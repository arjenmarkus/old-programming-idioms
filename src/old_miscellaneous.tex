\section{Miscellaneous topics}
Not all FORTRAN topics that might be useful to explain fit into the previous sections.
Therefore, this section describes some left-overs: Cray pointers, reading from and writing to
unopened files and so on.

\subsection{Cray pointers}
An extension that gained some popularity beyond the original compiler for which it was introduced
is the so-called Cray pointer. It is a device to work with C-like pointers in FORTRAN, enabling
it to take advantage of the memory address. A useful aspect is that it enables you to deal
with pointers to functions or subroutines in a way that was not possible with standard FORTRAN.

Here is a small example, take from the gfortran documentation \cite{F77Cray}:
\begin{verbatim}
      IMPLICIT NONE
      EXTERNAL SUB
      POINTER (SUBPTR,SUBPTE)
      EXTERNAL SUBPTE
      SUBPTR = LOC(SUB)
      CALL SUBPTE()
      [...]
      SUBROUTINE SUB
      [...]
      END SUBROUTINE SUB
\end{verbatim}

The statement \verb+POINTER( ... )+ links an integer \verb+SUBPTR+ to the address of a subroutine
\verb+SUBPTE+ (recognised as such by the \verb+EXTERNAL+ statement). It receives the actual
address of a subroutine \verb+SUB+ via the non-standard \verb+LOC()+ function. \verb+SUBPTE+ can then
be used instead of \verb+SUB+.

With this feature you could load a dynamic library (DLL, dylib or SO) and pass the pointers to the
routines around. With Fortran you can use procedure pointers instead, with the added benefit that
they can be given explicit signatures.

\subsection{Comparing logicals}
Logical variables have been part of the FORTRAN standard for a long time and they come with a full
set of logical operations. Comparing their values is done via the \verb+.EQV.+ or \verb+.NEQV.+ operations.
Some compilers, however, allow you to use \verb+.EQ.+ instead as well (see \verb+equiv.f+).

The following code is rejected by the gfortran compiler but accepted by Intel Fortran oneAPI:
\begin{verbatim}
      SUBROUTINE CHECK( A, B )
      LOGICAL A, B

      WRITE(*,*) 'A = ', A
      WRITE(*,*) 'B = ', B
      IF ( A .EQ. B ) THEN
          WRITE(*,*) 'A and B are equal'
      ELSE
          WRITE(*,*) 'A and B are NOT equal'
      ENDIF
\end{verbatim}

You should \emph{always} use \verb+.eqv.+ and \verb+neqv.+ for this sort of things, because of
portability but also because you do not know how the actual logical values are represented \cite{DrFortranEquiv}.


\subsection{Reading from and writing to unopened files}
Input and output in FORTRAN (and Fortran) is quite liberal: you do not have to open a file. IF you write to
a file that was not opened, then the program will create a file with a name like \verb+fort.88+, where 88
is the logical unit number you happen to write to. The actual name of the file that is thus created
depends on the compiler, so you cannot rely on this too much, especially when reading from a file
you did not open explicitly.

A very old compiler, the MicroSoft FORTRAN Powerstation compiler, would use a very different scheme:
it would use command-line arguments in the order in which they were needed in the program, as the
names for these unopened files.

As logical unit numbers are a shared, global, resource, it was not uncommon to get conflicts: a library
that wants to read or write to a file must acquire a unique number for its own use. Therefore, quite
often a routine would loop over a range of valid unit numbers to see which one was free (using the \verb+INQUIRE+
statement) and then claim it for its own:
\begin{verbatim}
      LOGICAL OPENED
      INTEGER I, LUN

      LUN = -1
      DO 110 I = 10,100
          INQUIRE( I, OPENED = OPENED )

          IF ( .NOT. OPENED ) THEN
              LUN = I
              GOTO 120
          ENDIF
  11  CONTINUE
  120 CONTINUE
\end{verbatim}

With the \verb+newunit=+ keyword to the \verb+open+ statement such constructions are no longer required.
It can be handy though to write debug output to some "anonymous" file, as you do not have to worry
about opening it only once and saving the unit number then.
