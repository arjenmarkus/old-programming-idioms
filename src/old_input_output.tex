\section{Input and output intricacies}
Placeholder:
\begin{verbatim}
- command-line arguments for file names
- big-endian and little-endian
- list-directed input and output - also: /
- narrow formats (?)
- use of d00 in input
- FORTRAN 66 semantics: OPEN - STATUS = 'NEW' as default.
- Effect of BLANK = 'ZERO' versus BLANK = 'NULL'
- 32-bits machines and unformatted files
\end{verbatim}

Almost any program will read some kind of input files and produce some kind of
output files. FORTRAN defined, roughly, three types of files:
\begin{itemize}
\item
Text files meant to read or edited by humans. These are known as \emph{formatted files}.
\item
Binary files with a record structure of sorts, so that you could read a part of the
record and then automatically jump to the next. These are \emph{unformatted sequential files}.
They are compact and might be considered \emph{binary} files.
\item
Binary files with records that have a fixed length and where you can position the \verb+READ+
or \verb+WRITE+ action to a particular record: the \emph{unformatted direct-access files}.\footnote{
Actually, there are also formatted direct-access files, but these are seldom used.}
\end{itemize}

Input and output in FORTRAN was always oriented towards records. For instance, if you read a number
from a formatted file, then the read position is automatically moved to the start of the next line,
independent of the amount of data left on the previous line.

Similarly, writing to a file always produced a complete line. And the next write action would start
on a new line.

Since Fortran 2003 the language also supports \emph{stream-access} for both unformatted and formatted
files. Before that standard, there were several more or less popular extensions to achieve the same
effect.


\subsection{Standard input and output}
FORTRAN had no concept of standard input and standard output files. But there was a de facto
standard in the form of logical unit numbers that were predefined:
\begin{itemize}
\item
The unit number \verb+6+ was typically connected to the terminal:
\begin{verbatim}
      WRITE(6,*) 'Enter a number:'
\end{verbatim}
\item
As an alternative you can also use the \verb+PRINT+ statement: it always writes to
standard ouput:
\begin{verbatim}
*     Write to the terminal ...
      PRINT(*) 'Current value: ', value
\end{verbatim}

\item
The unit number \verb+5+ was typically connected to the keyboard:
\begin{verbatim}
*     Let the user type a number and read it
      READ(5,*) value
\end{verbatim}
\item
Other "magic" unit numbers could exist, such as \verb+7+, often connected to a tape drive.
\item
Standard error could be connected to unit number \verb+0+ on some systems and would behave
differently than standard output: no buffering, not redirected to file without extra care.
\end{itemize}

The FORTRAN 77 standard introduced the asterisk as a convenient way to define standard input
and output without referring to a specific number: even though almost universal, these numbers
were not defined by the FORTRAN standard.

With the advent of Fortran~2008 the intrinsic module \verb+iso_fortran_env+ now defines
parameters like \verb+INPUT_UNIT+ to indicate the logical unit number for standard input and
similar ones for standard output and error.


\subsubsection{Carriage control}
Printers of old had a simple system to control the positioning of the output on the paper:
a character in the first column of a line to be output determined what would happen. Here
are the (common) codes:
\begin{table}[h!]
\begin{tabular}{ll}
\hline
Character    & Meaning \\
\hline
space        & Go to the next line -- the usual way of positioning \\
0            & Skip one line \\
1            & Go to the next page \\
+            & Do \emph{not} go to the next line -- print over the current \\
\end{tabular}
\end{table}

On some systems it would work on the terminal as well.

An extension to the write statement that had to do with controlling the cursor on a
terminal was the dollar sign:
\begin{verbatim}
      WRITE(6,'(a,$)') 'Enter a number:'
\end{verbatim}

The dollar sign meant that the cursor should stay on the last written line.

An alternative to the dollar sign was the backslash with the same meaning, at least
in compilers that supported that.


\subsection*{Positioning the pointer within a record}
Formats can be used to control the reading and writing of data in more ways than merely
specifying the numerical format or how many spaces to add. The "tab" edit descriptor
allows you to go to any position you like, including a previous position. Consider
this example:

\begin{verbatim}
     WRITE(6,'(A,T1,A)') 'Write a number', 'Read it'
\end{verbatim}

The "T1" edit descriptor set the position for the next item in an absolute fashion.
The statement writes the first string and then immediately afterwards overwrites the first
seven characters, so that the result is:

\begin{verbatim}
Read it number
\end{verbatim}

It is a feature that is not often encountered, but it is still part of the current standard.


\subsection{Direct-access files}
By default, direct-access files are unformatted -- values are dumped to the file or retrieved
without the help of a human-readable format. Properties of direct-access files:
\begin{itemize}
\item
You have to specify a record length, which holds for all records, when opening such a file.
You can, however, open the file with different record lengths, if your application benefits
from that. In other words, the length is not a property of the file itself.
\item
Direct-access files can be read or written by specifying a record number. Thus, like the name
implies, you can jump around in the file at will.
\item
The record length is usually specified in \emph{bytes} but some compilers, like Intel Fortran,
use \emph{words} as the unit. Where there was no way to know programmatically in FORTRAN what
size was meant, since Fortran 2003, the intrinsic module \verb+iso_fortran_env+ contains the parameter \verb+file_storage_unit+
which is the size in \emph{bits}.
\end{itemize}

Direct-access files, due to their simplicity, are compact and portable, as the structure of
unformatted sequential files depends on the compiler that was used for building the program (see below).

The main issues that make these files non-portable are the binary representation of the
numbers they contain. Nowadays, the main variation is the \emph{endianness}: the order of the
bytes that make up the number.

Here is a simple example of opening, writing and reading a direct-access file:

\begin{verbatim}
      PROGRAM DIRECTACCESS

      REAL VALUE(10)
      INTEGER I, REC

      OPEN( 10, FILE = 'directaccess.bin', ACCESS = 'DIRECT',
     &      RECL = 4*10 )

*
* CALCULATE SOME DATA AND WRITE THEM TO THE FILE
*
      DO 120 REC = 1,10
          DO 110 I = 1,10
              VALUE(I) = I + REC * 10.0
  110     CONTINUE

          WRITE( 10, REC = REC ) VALUE
  120 CONTINUE

*
* READ THE DATA - IN REVERSE ORDER
*
      DO 220 REC = 10,1,-1
          READ( 10, REC = REC ) VALUE(1), VALUE(2)
          WRITE( *, * ) VALUE(1), VALUE(2)
  220 CONTINUE
      END PROGRAM
\end{verbatim}

It produces output like:
\begin{verbatim}
   101.000000       102.000000
   91.0000000       92.0000000
   81.0000000       82.0000000
   71.0000000       72.0000000
   61.0000000       62.0000000
   51.0000000       52.0000000
   41.0000000       42.0000000
   31.0000000       32.0000000
   21.0000000       22.0000000
   11.0000000       12.0000000
\end{verbatim}


\subsection{Unformatted sequential files}
Data in sequential files, as the name suggests, are accessed in the order in which they appear in the files.
For unformatted files you write the records one by one and you can read the records back one by one.
But you cannot read more data from the record than its length. This is actually encoded in the file.
The following program will therefore fail, as it tries to read more data than present in the first
record:

\begin{verbatim}
      PROGRAM SEQUNFORM

      REAL VALUE(10)
      INTEGER I, J

      OPEN( 10, FILE = 'sequnform.bin', FORM = 'UNFORMATTED' )

*
* CALCULATE SOME DATA AND WRITE THEM TO THE FILE
* THE RECORDS GET LONGER
*
      DO 120 J = 1,10
          DO 110 I = 1,10
              VALUE(I) = I + J * 10.0
  110     CONTINUE

          WRITE( 10 ) (VALUE(I), I = 1,J )
  120 CONTINUE

*
* GOTO THE START AND THEN
* READ THE DATA - IN REVERSE ORDER
*
      REWIND( 10 )

*
* FIRST RECORD: ONLY ONE VALUE
*
      READ( 10 ) VALUE(1)
      WRITE( *, * ) VALUE(1)

*
* SECOND RECORD: TWO VALUES, BUT READ TEN
*
      READ( 10 ) VALUE
      WRITE( *, * ) VALUE(1)

      END PROGRAM
\end{verbatim}

Reading the first record works, but it fails on the second, as the program
tries to read 10 values (the whole array!), whereas the record only contain two
(the gfortran compiler was used):

\begin{verbatim}
   11.0000000
At line 38 of file sequnform.f (unit = 10, file = 'sequnform.bin')
Fortran runtime error: I/O past end of record on unformatted file

Error termination. Backtrace:

Could not print backtrace: libbacktrace could not find executable to open
#0  0xa11301fa
#1  0xa11278a1
#2  0xa1122d90
#3  0xa1148e3e
#4  0xa1130e81
#5  0xa1101841
#6  0xa11018e4
#7  0xa11013bd
#8  0xa11014f5
#9  0xb56c7613
#10  0xb76a26a0
#11  0xffffffff
\end{verbatim}

One essential difference between unformatted sequential files and stream-access files,
which were introduced in the Fortran~2003 standard, is that these unformatted sequential files
hold extra bytes to indicate the records. These extra bytes are an implementation detail,
not defined in the standard. A common scheme is:
\begin{itemize}
\item
The record starts with an integer number of four bytes that defines how many bytes
will follow.
\item
The bytes after that integer are the actual data.
\item
The record is closed with the same integer number, so that a check is possible but it
is also possible to go back in the file via the \verb+BACKSPACE+ statement\footnote{The
BACKSPACE statement originated with tapes as main mass data storage.}
\end{itemize}

Thus the following statement has the effect of skipping a record:
%
\begin{verbatim}
      READ(10)
\end{verbatim}
%
\noindent whereas with a stream-access file nothing would happen, as these files have no
concept of a record.

A common extension was that of the so-called \emph{binary} file. Such a file is essentially
what we now know as stream-access, but opening a file as \emph{binary} required a specific,
compiler-dependent keyword and not all compilers supported this type. Here is an example:
%
\begin{verbatim}
      OPEN(10, FILE = 'data.bin', FORM = 'BINARY')
\end{verbatim}


\subsection{Reading an array}
FORTRAN 77 defined a convenient way to read in arrays:
%
\begin{verbatim}
      REAL A(100)
      ...
      READ(10,*) A
\end{verbatim}
%
This reads 100 numbers from the file connected to unit number \verb+10+. But you can
also use the following method -- a so-called implied do-loop:
%
\begin{verbatim}
      INTEGER I, N
      REAL A(100)
      ...
      READ(10,*) N
      READ(10,*) (A(I), I = 1,N)
\end{verbatim}
%
A clever trick is to read the number of values and the values themselves in one statement:
%
\begin{verbatim}
      INTEGER I, N
      REAL A(100)
      ...
      READ(10,*) N, (A(I), I = 1,N)
\end{verbatim}
%
Because the value of N is known when the do-loop starts, there is no problem reading
the array in this fashion. (Unless, of course, N exceeds the size of the array.)
