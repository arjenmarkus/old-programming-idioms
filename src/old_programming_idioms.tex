\documentclass{article}

\date{\today}
\author{Arjen Markus}
\title{Old programming idioms explained}

\begin{document}
\maketitle

\section{Introduction}
In the more than 60 years of its existence the Fortran programming
language has undergone many changes, both in accordance with general insights
in programming paradigms and in reaction to developments in computer hardware.
This document discusses some of the idioms one finds in old FORTRAN packages
and their modern alternatives.

Please be aware that the use of lowercase and uppercase forms of the name is
not a whimsicality but rather marks a revolution within the language that was
started with the publication of the Fortran 90 standard in 1990(?). Throughout
this document we will use this distinction which is much more than merely
typographic.

The set-up is simple and ad hoc: we discuss various idioms that have been
used in the past decades and present contemporary equivalents or alternatives.
Attempts are made to present them in a systematic way, but that mostly means grouping
related topics.

\section{Source forms}
Placeholder:
\begin{verbatim}
- separate compilations, the misunderstanding of one routine per file
- compiler directives (especially proprietary ones)
\end{verbatim}


Directives: just an example, taken from the thread on MAX/MIN on the Intel Fortran forum
\begin{verbatim}
module m_mod
  implicit none
  real, allocatable, dimension (:) :: aa, bb, cc
  !dir$ attributes align:32 :: aa, bb, cc
  !dir$ assume_aligned aa:32, bb:32, cc:32
end module m_mod
\end{verbatim}

The source form of FORTRAN, now known as \emph{fixed form}, shows its heritage in the era of punchcards:
Each line could be up to 80 characters long, but only the first 72 characters had actual meaning. Some
editors would add line numbers in the column 73--80 or people could add short comments. It had and has
a few curiosities beyond the mere signficance of the columns.

The alternative source form that was introduced with the Fortran~90 standard is more flexible and should
definitely be used for any new source.

\subsection{The significance of spaces}
The original fixed form for FORTRAN sources has at least the following curiosities:
\begin{itemize}
\item
In the columns 1 to 5 you can only put comments, if the first character is a comment character ("C" or "*"),
or statement labels.
\item
The sixth column is reserved to indicate that the previous line is continued:
\begin{verbatim}
      WRITE(*,*) 'A long sentence that spans over two',
     &           'or more lines'
\end{verbatim}
The continuation character, in the above the "\&", can actually be any character with the exception of
a zero (0). Some programmers would use digits, so that you can easily count the number of lines.
\item
The columns 7 to 72 were used for the actual code.
\item
The columns 73 to 80 were reserved for comments, to the discretion of the user.
\item
\emph{Spaces had no significance}, except within literal strings.
\end{itemize}
Especially this last property may come as a surprise. To illustrate this (see \verb+fixedform.f+):
\begin{verbatim}
      PROGRAM FIXEDFORM

      INTEGER :: I

      DO 100 I = 1.10
          WRITE(*,*) I
  100 CONTINUE
!
!     ILLUSTRATE THE FACT THAT SPACES HAVE NO MEANING ...
!
      DO 110 I = 1,10
      I F ( I . E Q . 4 ) T H E N
          W R I T E ( * , * ) I
      E N D I F
  110 C O N T I N U E
      END
\end{verbatim}
Keywords in FORTRAN and Fortran are not reserved words and in fixed form you can
put between the letters as many spaces you want, anywhere.

Running this program might produce the following output:
\begin{verbatim}
   578772136
           4
\end{verbatim}
"Might", because there is an uninitialised variable here: the line \verb+DO 100 I = 1.100+
contains a typo. Instead of a comma, it contains a decimal point and thus the line is
interpreted as:
\begin{verbatim}
      DO100I = 1.10
\end{verbatim}
The variable \verb+DO100I+ gets assigned the value 1.10, which is why there is no DO loop
to produce ten lines in the output, counting from 1 to 10. Such mistakes can be caught
by using the \verb+IMPLICIT NONE+ statement, a common enough extension.\cite{Lionel}\footnote{This statement
was formalized in "MIL-STD-1753", precisely to make the coding safer.}

\emph{Note to self: https://stevelionel.com/drfortran/2021/09/18/doctor-fortran-in-implicit-dissent/}


\subsection{The character set}
Officially, FORTRAN code should be written with capitals only. Lowercase letters are only
allowed in literal strings. Again, a heritage of the machinery of old. But it has been
allowed by compilers to use lowercase as well.

Beyond letters of the Latin alphabet you can use digits and underscores and several other characters.
Even today, the standard is quite strict: characters outside the officially supported set are only
allowed (or tolerated?) in literal strings.


\subsection{Variables, functions, names and types}
A feature that Fortran code relies on less and less is the use of \emph{implicit typing}: of old
a variable, or indeed a function, was of type \verb+integer+ if its name starts with one of: I, J, K, L, M or N.
If not, the variable or function was of type \verb+real+. You could declare the name as being of a different type, but
required an explicit declaration.

Because of this implicit typing, mistakes in names were a serious source of errors.

In FORTRAN, names were also limited to six (significant) characters, a limitation shared with the linkers.

This six-characters limit was even more severe as the names of subroutines, functions and \verb+COMMON+ blocks
were global for the whole program. There was no other scope or control of visibility. This made it
very important to explicitly define the names of such entities for any library you wrote, as naming conflicts
could not be solved via a renaming clause, as they can nowadays in Fortran.


\subsection{Special comments: D}
Sometimes in old code you can see a character "D" in the first column. This is considered a comment line,
unless you specify a compile option like \verb+/d-lines+ for the Intel Fortran compilers. Not all compilers
support this, gfortran does not seem to have an option for it. The effect of specifying the relevant
option is that the line is no longer considered a comment line, but is actual code. If supported by the compiler,
it is only available for fixed form source.

It is a rather awkward form of "preprocessing":
\begin{itemize}
\item
It is a compiler extension or at least not supported by all compilers. The now deprecated g77 compiler acknowledged
the existence of such lines, but provided no facilities.

\emph{Note to self: https://gcc.gnu.org/onlinedocs/gcc-3.4.6/g77/Debug-Line.html}
\item
It is not available for free form source.
\item
A practical problem is that it does not evolve with the code itself, as in the normal build process, these
lines are considered comments.
\end{itemize}

It is just as easy to use ordinary code to provide debugging facilities:
\begin{verbatim}
      PROGRAM FIXEDDEBUG

D     WRITE(*,*) 'Note: in debugging mode ...'
      WRITE(*,*) 'Hello, world'

      END PROGRAM FIXEDDEBUG
\end{verbatim}

The program (see the file \verb+fixeddebug.f+) is not accepted by gfortran because of the D-line, but Intel Fortran
compiles it with and without the option \ver+-d-lines+, only producing different output.

You could make the intent clearer using something along these lines (or a modern equivalent):
\begin{verbatim}
      PROGRAM FIXEDDEBUG
      LOGICAL DEBUG
      PARAMETER (DEBUG = .FALSE.)

      IF (DEBUG) WRITE(*,*) 'Note: in debugging mode ...'
      WRITE(*,*) 'Hello, world'

      END PROGRAM FIXEDDEBUG
\end{verbatim}

With FORTRAN the \verb+debug+ parameter was best put in an include file to ensure that every programn unit
used the same setting. With Fortran a module containing such a parameter will do.

\quote{
\emph{Note:} If you use a \emph{parameter} instead of a \emph{variable}, many compilers will even eliminate
the debugging code from the resulting program, but the code is compiled.}

\section{Subroutines and functions}
Placeholder:
\begin{verbatim}
- array(*) versus array(:)
- array(10) as the starting point
- constants as actual arguments
- intent
- temporary arrays - non-contiguous arrays
- checking interfaces
- entry
- external, also notation "*tan"
- Specific names for functions like max() and sin()
- Alternate return
- array(1) instead of array(*)
- statement functions

\end{verbatim}

\section{Floating-point numbers}
Nowadays, most computers use the IEEE format for representing floating-point numbers.
The two main types you will encounter are single-precision reals, occupying four
bytes of memory, and double-precision reals, occupying eight bytes. Fortran of old,
including FORTRAN, favours single-precision -- without any decoration a literal
number like \verb+1.23456789+ is single-precision, whereas many other languages
use double-precision reals by default.

Back in the days before the IEEE standard was widely adopted \cite{IEEE}, reals were
represented in many different ways and also the arithmetic operations we normally take
for granted were not fully portable. This led to all manner of complications if
you wanted to make your program portable, and that was and is certainly a goal of Fortran.

\subsection{REAL*4 and REAL*8}
One of the many extensions that were added by compiler vendors was the use of an asterisk~(*)
to indicate the precision:
\begin{verbatim}
*
*     Single precision real
*
      REAL*4 X
*
*     DOUBLE PRECISION REAL
*
      REAL*8 Y
*
*     Single precision complex
*
      COMPLEX*8 Z
\end{verbatim}
The number indicates the number of bytes that a real would occupy. This has never been
part of a FORTRAN or Fortran standard. The \emph{kind} feature in Fortran is much more
flexible, as it can capture aspects of the representation of floating-point numbers
beyond mere storage.\footnote{The current standard defines a general model for representing
real numbers. This encompasses the IEEE formats, but is in fact more general.}

While FORTRAN has supported complex numbers for a very long time, it did not standardise
double-precision complex numbers. Compiler extensions like \verb+COMPLEX*16+ often
filled that gap. With Fortran, the \emph{kind} feature is used to achieve this:
\begin{verbatim}
integer, parameter :: dp = kind(1.d0)
complex(kind=dp)   :: z
\end{verbatim}

The kind refers to the underlying floating-point number, not the storage, like with
the \verb+COMPLEX*16+ type.

\begin{quote}
\emph{Note:} Such notation is sometimes used for integers and logicals as well. Again
the \emph{kind} feature is much more useful than merely indicating the storage size.
\end{quote}

\begin{quote}
\emph{Note:} I have used a Convex computer in the distant past that actually had two
different types of floating-point numbers that both were single-precision. One was
structured according to the IEEE standard and the other was a native format. The difference
for all intents and purposes was the interpretation of the exponent. The native format
was said to be a bit faster, but they occupied the same storage, four bytes.

For the same bit pattern the \emph{values} differed by a factor~4.
\end{quote}

\subsection{Literals in the source code}
One thing to keep in mind: if a literal number occurs in the source code, it is interpreted
as it appears, independent of the context. For Fortran this has been standardised: an expression
on the right-hand side is evaluated independently of the left-hand side. More concretely:
\begin{verbatim}
    double precision pi = 3.14159265358979323846264338327950288419716939937510
\end{verbatim}
\noindent may look to specify $\pi$ in some 50 decimals, but to the compiler it is
merely a slightly bizarre way of expressing it in \emph{single-precision}, so actually
only six or seven significant decimals. To get \emph{double-precision}, you need to
add a \emph{kind} or, as it was in FORTRAN, a "d" exponent (with some excess decimals removed):
\begin{verbatim}
    double precision pi = 3.141592653589793238462643d0
\end{verbatim}

You can see the difference if you run this program:
\begin{verbatim}
program diff_double_precision
    implicit none

    double precision, parameter :: pi_1 = 3.141592653589793238462643
    double precision, parameter :: pi_2 = 3.141592653589793238462643d0

    write(*,*) 'Difference: ', pi_1 - pi_2
end program diff_double_precision
\end{verbatim}
\noindent which prints (you may expect slight differences in the last few decimals with
different compilers):
\begin{verbatim}
 Difference:    8.7422780126189537E-008
\end{verbatim}

Some old FORTRAN compilers seem to have been less strict about the dichotomy between
the left-hand side and the right-hand side and would indeed interpret such literal
numbers as double-precision.

Another thing to keep in mind is that many compilers, both new and old, allow for
compiler options that turn the \emph{default} precision for a variable declared as \verb+real+
into \emph{double precision}. If a program relies on this behaviour, then you need to
carefully check the code.\footnote{In general, using these compile options is
considered a bad idea. It is all too easy to forget them when building the program or library.}


\subsection{Retrieving properties of the floating-point arithmetic}
Quite often in robust numerical algorithms, like calculating special functions (Bessel, gamma, ...), it is necessary
to know the properties of the floating-point system the program uses, such as the smallest
or the largest value that can be represented. In Fortran there are various intrinsic functions
that allow this information to be queried, but in FORTRAN that was not standardised. Also,
the floating-point systems differed a great deal more than they do nowadays. To solve that,
functions called \verb+R1MACH+, \verb+D1MACH+ and even \verb+I1MACH+ were used that returned
the required information. They sometimes had to be adapted to the machine you were working
on, as indicated in the comments of \verb+R1MACH+:

\begin{verbatim}
      REAL FUNCTION R1MACH(I)
      INTEGER I
C
C  SINGLE-PRECISION MACHINE CONSTANTS
C  R1MACH(1) = B**(EMIN-1), THE SMALLEST POSITIVE MAGNITUDE.
C  R1MACH(2) = B**EMAX*(1 - B**(-T)), THE LARGEST MAGNITUDE.
C  R1MACH(3) = B**(-T), THE SMALLEST RELATIVE SPACING.
C  R1MACH(4) = B**(1-T), THE LARGEST RELATIVE SPACING.
C  R1MACH(5) = LOG10(B)
C
      INTEGER SMALL(2)
      INTEGER LARGE(2)
      INTEGER RIGHT(2)
      INTEGER DIVER(2)
      INTEGER LOG10(2)
C     needs to be (2) for AUTODOUBLE, HARRIS SLASH 6, ...
      INTEGER SC
      SAVE SMALL, LARGE, RIGHT, DIVER, LOG10, SC
      REAL RMACH(5)
      EQUIVALENCE (RMACH(1),SMALL(1))
      EQUIVALENCE (RMACH(2),LARGE(1))
      EQUIVALENCE (RMACH(3),RIGHT(1))
      EQUIVALENCE (RMACH(4),DIVER(1))
      EQUIVALENCE (RMACH(5),LOG10(1))
      INTEGER J, K, L, T3E(3)
      DATA T3E(1) / 9777664 /
      DATA T3E(2) / 5323660 /
      DATA T3E(3) / 46980 /
C  THIS VERSION ADAPTS AUTOMATICALLY TO MOST CURRENT MACHINES,
C  INCLUDING AUTO-DOUBLE COMPILERS.
C  TO COMPILE ON OLDER MACHINES, ADD A C IN COLUMN 1
C  ON THE NEXT LINE
      DATA SC/0/
C  AND REMOVE THE C FROM COLUMN 1 IN ONE OF THE SECTIONS BELOW.
C  CONSTANTS FOR EVEN OLDER MACHINES CAN BE OBTAINED BY
C          mail netlib@research.bell-labs.com
C          send old1mach from blas
C  PLEASE SEND CORRECTIONS TO dmg OR ehg@bell-labs.com.
C
C     MACHINE CONSTANTS FOR THE HONEYWELL DPS 8/70 SERIES.
C      DATA RMACH(1) / O402400000000 /
C      DATA RMACH(2) / O376777777777 /
C      DATA RMACH(3) / O714400000000 /
C      DATA RMACH(4) / O716400000000 /
C      DATA RMACH(5) / O776464202324 /, SC/987/
       ...
\end{verbatim}

These functions were carefully constructed as were the algorithms that used
them. In this day and age, almost all such variation in the properties of
floating-point numbers has been eliminated via the adoption of the IEEE standard.
That does not mean of course that floating-point arithmetic holds no surprises
anymore. (One classic text on the subject is


\subsection{Input in the absence of a decimal point}
Disk storage nowadays is all but endless, but this luxury did not exist in the old days.
This may have been the reason for a little known or used feature in the input of real numbers:
if a string representing a real number does not contain a decimal point, then the \emph{input~format}
may insert it.

Here is an example, using internal I/O to make it self-contained (see also the file \verb+input_no_point.f90+):
\begin{verbatim}
program show_insert_point
    implicit none

    real :: x
    character(len=10) :: string

    string = '1234'

    read( string, '(f4.0)' ) x
    write(*,*) x

    read( string, '(f4.2)' ) x
    write(*,*) x
end program show_insert_point
\end{verbatim}
It produces:
\begin{verbatim}
   1234.00000
   12.3400002
\end{verbatim}
With the format in the second read statement a decimal point is inserted!

It may have been useful in the past but it does suggest that to avoid surprises, you better not
use input format with a prescribed number of decimals.



\section{Control structures}
Placeholder:
\begin{itemize}
\item
Nested do-loop
\item
Simulated do-while
\item
Jumping out of a do-loop
\item
IF-constructions, including "select-case"
\end{itemize}

\section{Memory management}

In the decades leading up to the FORTRAN 77 standard, memory management was simple:
declare what memory you need via statically sized arrays and that is it. There was
no dynamic memory allocation, at least not in the FORTRAN language. It may surprise
you, but even the concept of an operating system that took care of the computer
was fairly new, as illustrated by a 1971 book by D.W. Barron, titled "Computer
Operating Systems". Quoting from the book's jacket:
\begin{quote}
As the operating system is becoming an important part of the software complex
accompanying a computer system. A large amount of knowledge about the subject
now exists, mainly in the form of papers in computer journals. It is thus time for
a book that coordinates what is known about operating systems.
\end{quote}

There are, however, a few aspects of FORTRAN that make the story a bit more
complicated: \verb+COMMON+ blocks, \verb+EQUIVALENCE+ and the \verb+SAVE+ statement.
All three will be discussed here.


\subsection{COMMON blocks}
Variables, be they scalars or arrays, are normally passed via argument lists between
program units (the main program, subroutines or functions). This is the immediately
visible part. But you can also pass variables via \verb+COMMON+ blocks. These
constitute a form of global \emph{memory}, but not of global \emph{variables}, as
a \verb+COMMON+ block merely allocates memory and the mapping of memory
locations onto variables is up to the program units themselves. For instance:
\begin{verbatim}
      SUBROUTINE SUB1
      COMMON /ABC/ X(10)
      ...
      END

      SUBROUTINE SUB2
      COMMON /ABC/ A(5), B(5)
      ...
      END
\end{verbatim}
The \verb+COMMON+ block \verb+/ABC/+ appears in two subroutines, but in subroutine
\verb+SUB1+ it is associated with the array \verb+X+ of 10 elements and in the
subroutine \verb+SUB2+ it is associated with two arrays, \verb+A+ and \verb+B+,
both having five elements. The memory is shared, so that if you set \verb+X(1)+
to, say, \verb+1.1+ in subroutine \verb+SUB1+, then on the next call to subroutine
\verb+SUB2+, the array element \verb+A(1)+ will have that same value, as they occupy
the same memory location.

\verb+COMMON+ blocks should have the same \emph{size} in all locations in the program's
code where they occur. That is difficult to ensure, hence it was common (no pun intended)
to put the declaration of \verb+COMMON+ blocks in so-called include files. Each program
unit that needed to address the memory allocated via these \verb+COMMON+ blocks could
then use the \verb+INCLUDE+ statement to have the compiler insert the literal text
of that include file.\footnote{The INCLUDE statement was actually a common compiler
extension.}

There are various ways that \verb+COMMON+ blocks were used:
\begin{itemize}
\item
Variables in \verb+COMMON+ blocks are persistent. At least, that was a very common
occurrence. The rules in the FORTRAN standard are more complicated, but certainly
with the \verb+SAVE+ statement you can rely on these variables to retain the
values between calls to a routine.
\item
Often routines in a library have to cooperate: one routine is used to set options
and other routines do the actual work. By using one or more \verb+COMMON+ blocks
these options do not need to be passed around via the argument list.
\item
Together with \verb+EQUIVALENCE+ statements you could use the \verb+COMMON+ blocks
to share workspace. Remember: back in the days memory was much and much more precious
and scarcer than it is now. So, defining work arrays \verb+WORK+ (of type real)
and \verb+IWORK+ (of type integer) and making them equivalent to each other, you
could save on memory, if these arrays are not used at the same time.

In the code that would look like:
\begin{verbatim}
      COMMON /ABC/ WORK(1000)
      EQUIVALENCE (WORK(1), IWORK(1))
\end{verbatim}
\noindent with a typical sloppiness with respect to array dimensions.

\end{itemize}
Nowadays, it is much easier to pass large amounts of essentially private data
around, simply define a suitable derived type. Also, it is easy to allocate
work arrays as you require them and release them again when done.

A special \verb+COMMON+ block was the so-called \emph{blank} \verb+COMMON+
block. It had no name and it did not have to be declared with the same size
in all parts of the program. In fact, on some systems it could be used as
a flexible reservoir of memory, in much the same way as you have the
heap nowadays. But this particular use was an extension to the standard.


\subsubsection{More on EQUIVALENCE}
A specific use of the \verb+EQUIVALENCE+ statement is to access
the binary representation of a (floating-point) number. A concrete
example is the conversion of big-endian numbers to little-endian
numbers or vice versa:
%
\begin{verbatim}
      INTEGER      A, B
      CHARACTER*4  BIG, LITTLE
      EQUIVALENCE (A, BIG)
      EQUIVALENCE (B, LITTLE)
C
C     Read some number from a file - big endian
C     and rearrange the bytes
C
      READ(10) A
      LITTLE(1:1) = BIG(4:4)
      LITTLE(2:2) = BIG(3:3)
      LITTLE(3:3) = BIG(2:2)
      LITTLE(4:4) = BIG(1:1)

      WRITE(*,*) 'Value is: ', B
\end{verbatim}

With Fortran, however, you can use the \verb+transfer()+ function to
convert the integer to a sequence of four individual characters or
simply use the bit manipulation functions to extract the information.


\subsection{The SAVE statement}
According to the FORTRAN standard a local variable in a function or
subroutine does not retain its value between calls, unless it has the
\verb+SAVE+ attribute:
%
\begin{verbatim}
      SUBROUTINE ACCUM( ADD )
*
*     Accumulate the counts
*
      INTEGER ADD

      INTEGER TOTAL
      SAVE    TOTAL
      DATA    TOTAL / 0 /

      TOTAL = TOTAL + ADD
      IF ( TOTAL > 100 ) THEN
          WRITE(*,*) 'Reached: ', TOTAL
      ENDIF
      END
\end{verbatim}

However, some implementations, notably on DOS/Windows, used static
storage for these local variables, which meant that the variables would
\emph{seemingly} retain their values, even without the \verb+SAVE+ statement.
If a program relied on this property and was ported to a different environment,
all manner of havoc could be raised.

\begin{quote}
\emph{Note:} I have actually had lively, but not necessarily pleasant, debates on
whether the behaviour either way was correct. Sometimes the unexpected
behaviour was claimed to indicate a compiler bug.

Some compilers have an option to enforce the \verb+SAVE+ attributes on variables,
irrespective of the source code. You should take special care if old source code
relies on such an option.
\end{quote}

\subsection{The initial values of (local) variables}
A feature related in a way to the \verb+SAVE+ statement is the fact that
in both FORTRAN and Fortran variables do not get a particular initial
value, unless they have the \verb+SAVE+ attribute, implicitly or explicitly.
With older compilers local variables may be stored in static memory and
quite often they may have an initial value of zero or whatever the
equivalent is for the variable's type, but that is in all cases simply
a random circumstance. \emph{Never assume that a variable that has not
been explicity given a value, has a particular value.}

You can set the initial value in FORTRAN via the \verb+DATA+ statement:
%
\begin{verbatim}
      LOGICAL FIRST
      DATA FIRST / .TRUE. /
\end{verbatim}

This means that at the first call to the subroutine holding this variable
\verb+FIRST+, it has the value \verb+.TRUE.+. You can later set it to
\verb+.FALSE.+ to indicate that the subroutine has been called at least once
before, so that no initialisation is needed anymore:
%
\begin{verbatim}
* Subroutine that sums the values we pass
      SUBROUTINE SUM( x )
      INTEGER X

      INTEGER TOTAL
      LOGICAL FIRST
      DATA FIRST / .TRUE. /

      IF ( FIRST ) THEN
          FIRST = .FALSE.
          TOTAL = 0
      ENDIF

      TOTAL = TOTAL + X
      END
\end{verbatim}
(Just a variation on the previous example). This is not a very interesting routine,
but it illustrates a typical use.

\emph{To emphasize:} This type of initialisation is done so that the variables
in question have the designated value at the first call. If you change
the value, then they retain that new value. No reinitialisation occurs.
(Actually, the value is not set on the first call, but rather is part
of the data section of the program as a whole. There is no separate
assignment.)

The \verb+DATA+ statement is not executable, it normally appears
somewhere in the section that defines the variables, but it may occur
elsewhere -- most FORTRAN and Fortran compilers are not strict about it.

There is some peculiar syntax involved:
%
\begin{verbatim}
      REAL X(100)
      DATA X / 1.0, 98*0.0, 100.0 /
\end{verbatim}

You can repeat values in a somewhat similar way as with edit descriptors
in format statements: a count followed by an asterisk (*) and the value
to be repeated. It is also possible to use implied do-loops:
%
\begin{verbatim}
      INTEGER I
      REAL X(100)

      DATA (X(I), I = 1,100) / 100*1.0 /
\end{verbatim}

While it is more usual to set the values together with the declaration
of a variable nowadays, like:
%
\begin{verbatim}
  integer :: i
  real    :: x(100) = [ (1.0, i = 1,size(x)) ]
\end{verbatim}
\noindent the \verb+DATA+ statement is more versatile, because it is
not necessary to set the values for an array in one single statement:
%
\begin{verbatim}
    integer :: i
    real    :: x(100)

    data (x(i), i = 1,50)   / 50*1.0 /
    data (x(i), i = 51,100) / 50*0.0 /
\end{verbatim}

So, this old-fashioned statement may have its uses still.

Another peculiarity: the \verb+DATA+ statement has effect on the
size of the object file and thus the executable itself. The following
program leads to an executable of approximately 5.7 MB using gfortran
on Windows:
\begin{verbatim}
program data_stmt
    implicit none

    integer :: i
    real    :: array(1000000)

    data (array(i), i = 1,size(array),2) / 500000*1.0 /
    data (array(i), i = 2,size(array),2) / 500000*2.0 /

    !
    ! Alternative
    !
    !! array(1::2) = 1.0
    !! array(2::2) = 2.0
    !

    write(*,*) sum(array)
end program data_stmt
\end{verbatim}

If, instead, you use the alternative and remove the \verb+DATA+
statements, the executable is only 1.7 MB. Of course, this is
an exaggerated example, but it illustrates that such \verb+DATA+
statements are very different in character than ordinary, executable
statements.


\subsection{Initialising variables in COMMON blocks: BLOCK DATA}
\label{blockdata}
The \verb+DATA+ statement plays an important role when it comes to initialising
variables in a \verb+COMMON+ block. Since the \verb+COMMON+ blocks usually
appear in more than one subprogram (main program, subroutines, functions),
they cannot be initialised in the same way as ordinary variables: which
\verb+DATA+ statement should prevail, if several initialise the same
\verb+COMMON+ variables?

Thus enter the \verb+BLOCK DATA+ program unit!

It is the only way to initialise variables in a \verb+COMMON+ block and
it is special, because it is not executable and is not part of a routine
or the main program. The peculiar consequence is
that you cannot put it in a library: there is no reference to it, unlike
with subroutines and functions, so it would never be loaded. Instead, you
will normally put in the same file as the main program or link against
its object file explicitly.

The general layout is:
%
\begin{verbatim}
      BLOCK DATA
      ... COMMON blocks ...
      ... DATA statements ...
      END
\end{verbatim}

Here is a small example of this effect:\footnote{I use the gfortran compiler
to illustrate such effects, but it would be similar with other compilers.
And since most if not all FORTRAN features are still supported in Fortran,
I also use free-form sources.}
%
\begin{verbatim}
gfortran -o common1 common.f90 block.f90
gfortran -o common2 common.f90
\end{verbatim}

Both build commands succeed, despite the fact that one part of the program is missing
in the second one.

The \verb+block.f90+ source file is:
%
\begin{verbatim}
BLOCK DATA
COMMON /ABC/ X
DATA X /42/
END
\end{verbatim}

The \verb+common.f90+ source file is:
%
\begin{verbatim}
PROGRAM PRINTX
COMMON /ABC/ X
WRITE(*,*) 'Expected value of X = 42:'
WRITE(*,*) 'X = ', X
END
\end{verbatim}

Program \verb+common1+ prints the value 42, whereas the other
program prints 0. If the \verb+BLOCK DATA+ program unit had been
an ordinary program unit, the building of this version would have failed
on an unresolved symbol or the like.


\subsection{Work arrays}
\label{workarrays}
In the old days you could encounter arguments to a routine that represented
such workspace. Usually you would have to declare the arrays to a size that
matches the problem at hand. Here is an example from the \emph{LAPACK} library
for linear algebra:
%
\begin{verbatim}
      SUBROUTINE DGELS( TRANS, M, N, NRHS, A, LDA, B, LDB, WORK, LWORK,
     $                  INFO )
*
*  -- LAPACK driver routine (version 3.2) --
*  -- LAPACK is a software package provided by Univ. of Tennessee,    --
*  -- Univ. of California Berkeley, Univ. of Colorado Denver and NAG Ltd..--
*     November 2006
*
*     .. Scalar Arguments ..
      CHARACTER          TRANS
      INTEGER            INFO, LDA, LDB, LWORK, M, N, NRHS
*     ..
*     .. Array Arguments ..
      DOUBLE PRECISION   A( LDA, * ), B( LDB, * ), WORK( * )
*     ..
\end{verbatim}

In this case, the argument \verb+WORK+ is a double-precision array of
size \verb+LWORK+. In the comments that document the use of this routine
the precise usage is described:
%
\begin{verbatim}
*  WORK    (workspace/output) DOUBLE PRECISION array, dimension (MAX(1,LWORK))
*          On exit, if INFO = 0, WORK(1) returns the optimal LWORK.
*
*  LWORK   (input) INTEGER
*          The dimension of the array WORK.
*          LWORK >= max( 1, MN + max( MN, NRHS ) ).
*          For optimal performance,
*          LWORK >= max( 1, MN + max( MN, NRHS )*NB ).
*          where MN = min(M,N) and NB is the optimum block size.
*
*          If LWORK = -1, then a workspace query is assumed; the routine
*          only calculates the optimal size of the WORK array, returns
*          this value as the first entry of the WORK array, and no error
*          message related to LWORK is issued by XERBLA.
\end{verbatim}

This means that for a particular case you can either use one of the formulae
or the special value \verb+-1+ for \verb+LWORK+ to obtain an optimal value.
The work array itself would still be a statically declared array.

Note that with the current features of Fortran the interface could be greatly
simplified:\footnote{Intentionally left in fixed form.}
%
\begin{verbatim}
      SUBROUTINE DGELS( TRANS, A, B, INFO )
*
*     .. Scalar Arguments ..
      CHARACTER, INTENT(IN) :: TRANS
      INTEGER, INTENT(OUT)  :: INFO
*     ..
*     .. Array Arguments ..
      DOUBLE PRECISION, INTENT(INOUT) ::  A(:,:), B(:,:)
*     ..
\end{verbatim}
%
\noindent provided the interface is made explicit via a module or an interface block.

\begin{quote}
The careful reader wil note that one feature of the original interface
has not been retained in the simplification: \verb+NHRS+. Thus, with this
revised interface the array \verb+B+ should consist entirely of right-hand
side vectors.
\end{quote}

\section{Input and output intricacies}
Placeholder:
\begin{verbatim}
- standard input and output
- LU-numbers 5 and 6 (and 7)
- command-line arguments for file names
- big-endian and little-endian
- unformatted versus binary files
- list-directed input and output - also: /
- narrow formats (?)
- use of d00 in input
- FORTRAN 66 semantics: OPEN - STATUS = 'NEW' as default.
- Effect of BLANK = 'ZERO' versus BLANK = 'NULL'
- read(10,*) n, (array(i), i = 1,n)
- 32-bits machines and unformatted files
- direct-access files
- carriage control: 0, 1, +
\end{verbatim}





\section{Subjects}
\begin{verbatim}
- array(*) versus array(:)
- array(10) as the starting point
- history of computers:
   - hardware
   - memory management
   - tools like source code control systems
   - connections between computers, Internet
- equivalence
- constants as actual arguments
- intent
- temporary arrays - non-contiguous arrays
- implicit types
- double precision versus kind
- checking interfaces
- separate compilations, the misunderstanding of one routine per file
- fixed form and spaces
- standard input and output
- LU-numbers 5 and 6 (and 7)
- command-line arguments for file names
- real do-variables
- entry
- statement functions
- six characters
- numerical binary representations versus IEEE (IBM, Cray, Convex)
- big-endian and little-endian
- double complex
- unformatted versus binary files
- list-directed input and output - also: /
- narrow formats (?)
- use of d00 in input
- Cray pointers
- uppercase/lowercase letters
- D as comment character
- external, also notation "*tan"
- FORTRAN 66 semantics: OPEN - STATUS = 'NEW' as default.
- Effect of BLANK = 'ZERO' versus BLANK = 'NULL'
- Specific names for functions like max() and sin()
- Alternate return
- Equating logicals: logical x, y; if ( x .eq. y ) then
\end{verbatim}


References to be added: IEEE 754 and Fortran 90 standard.

\end{document}
