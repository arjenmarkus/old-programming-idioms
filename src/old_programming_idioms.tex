\documentclass{article}

\date{\today}
\author{Arjen Markus}
\title{Old programming idioms explained}

\begin{document}
\maketitle

\section{Introduction}
In the more than 60 years of its existence the Fortran programming
language has undergone many changes, both in accordance with general insights
in programming paradigms and in reaction to developments in computer hardware.
This document discusses some of the idioms one finds in old FORTRAN packages
and their modern alternatives.

Please be aware that the use of lowercase and uppercase forms of the name is
not a whimsicality but rather marks a revolution within the language that was
started with the publication of the Fortran 90 standard in 1990(?). Throughout
this document we will use this distinction which is much more than merely
typographic.

The set-up is simple and ad hoc: we discuss various idioms that have been
used in the past decades and present contemporary equivalents or alternatives.
Attempts are made to present them in a systematic way, but that mostly means grouping
related topics.

\section{Source forms}
The source form of FORTRAN, now known as \emph{fixed form}, shows its heritage in the era of punchcards:
Each line could be up to 80 characters long, but only the first 72 characters had actual meaning. Some
editors would add line numbers in the column 73--80 or people could add short comments. It had and has
a few curiosities beyond the mere signficance of the columns.

The alternative source form that was introduced with the Fortran~90 standard is more flexible and should
definitely be used for any new source.


\subsection{Fixed form versus free form}
FORTRAN source files usually have an extension \verb+.f+ and on PCs, before you could use long names,
the extension often was \verb+.for+. There is, however, nothing special about these file extensions. They
are merely a convention, useful for the compiler as it can use it to identify the source form. The
free form source is often indicated by the extension \verb+.f90+. \emph{This should not be taken to mean
that the code adheres to the Fortran~90 standard}, just as there is no particular difference as far as
file extensions are concerned for FORTRAN~66 or FORTRAN~77.\footnote{File extensions themselves are
a fairly recent feature of computer file systems. Older systems had different conventions.}

To elaborate a bit on these file extensions:
\begin{itemize}
\item
Fixed form is often identified by \verb+.f+ or for some compilers on Windows \verb+.for+, but
that is not mandated by any standard and compilers often support options to specify what the
source form is, if the extension is misleading.
\item
On file systems where file names are case-sensitive, such as Linux, an extension \verb+.F+ or \verb+.F90+
is often meant to automatically invoke a preprocessor like C's preprocessor. Alternatively, you
can also use compiler options to invoke the preprocessor explicitly.
\item
Preprocessing is not part of the FORTRAN or Fortran standards. It is simply an extension (no pun intended)
that may or may not be supported by the compiler. As a consequence, there is no prescribed syntax, though
very often the C conventions are used.
\end{itemize}


\subsection{The significance of spaces}
The original fixed form for FORTRAN sources has at least the following curiosities:
\begin{itemize}
\item
Any non-blank character in the first column indicates that the line is a comment.
Common conventions were to use the characters "!", "C", or "*".
\item
Columns 2 through 5 are reserved for statement labels.
\item
Any non-blank character in the sixth column (other than "0") indicates that the previous line is continued:
\begin{verbatim}
*234567890
      WRITE(*,*) 'A long sentence that spans over two',
     &           'or more lines'
\end{verbatim}
The most common convention was to use the "\&". The "*" was not uncommon either.
Some programmers would use digits, so that you can easily count the number of lines.
\item
The columns 7 to 72 were used for the actual code.
\item
The columns 73 to 80 were reserved for comments, to the discretion of the user.
\item
\emph{Spaces had no significance}, except within literal strings.
\end{itemize}

Especially this last property may come as a surprise. To illustrate this (see \verb+fixedform.f+):
\begin{verbatim}
      PROGRAM FIXEDFORM

      INTEGER :: I

      DO 100 I = 1.10
          WRITE(*,*) I
  100 CONTINUE
!
!     ILLUSTRATE THE FACT THAT SPACES HAVE NO MEANING ...
!
      DO 110 I = 1,10
      I F ( I . E Q . 4 ) T H E N
          W R I T E ( * , * ) I
      E N D I F
  110 C O N T I N U E
      END
\end{verbatim}
Keywords in FORTRAN and Fortran are not reserved words and in fixed form you can
put between the letters as many spaces you want, anywhere.

Running this program might produce the following output:
\begin{verbatim}
   578772136
           4
\end{verbatim}
"Might", because there is an uninitialised variable here: the line \verb+DO 100 I = 1.100+
contains a typo. Instead of a comma, it contains a decimal point and thus the line is
interpreted as:
\begin{verbatim}
      DO100I = 1.10
\end{verbatim}
The variable \verb+DO100I+ gets assigned the value 1.10, which is why there is no DO loop
to produce ten lines in the output, counting from 1 to 10. Such mistakes can be caught
by using the \verb+IMPLICIT NONE+ statement, a common enough extension.\cite{drFortranImplicit}\footnote{This statement
was formalized in "MIL-STD-1753", precisely to make the coding safer.}


\subsection{The character set}
Officially, FORTRAN code should be written with capitals only. Lowercase letters are only
allowed in literal strings. Again, a heritage of the machinery of old. But it has been
allowed by compilers to use lowercase as well.

Beyond letters of the Latin alphabet you can use digits and underscores and several other characters.
Even today, the standard is quite strict: characters outside the officially supported set are only
allowed (or tolerated?) in literal strings.


\subsection{Variables, functions, names and types}
A feature that Fortran code relies on less and less is the use of \emph{implicit typing}: of old
a variable, or indeed a function, was of type \verb+integer+ if its name starts with one of: I, J, K, L, M or N.
If not, the variable or function was of type \verb+real+. You could declare the name as being of a different type, but
that required an explicit declaration.

Because of this implicit typing, mistakes in names were a serious source of errors.

In FORTRAN, names were also limited to six (significant) characters, a limitation shared with the linkers.

This six-characters limit was even more severe as the names of subroutines, functions and \verb+COMMON+ blocks
were global for the whole program. There was no other scope or control of visibility. This made it
very important to explicitly define the names of such entities for any library you wrote, as naming conflicts
could not be solved via a renaming clause, as they can nowadays in Fortran.

While the \verb+IMPLICIT+ statement is used nowadays mostly to turn off the implicit typing via \verb+implicit none+,
it can be used to define a particular type to variable whose names started with a particular character:
\begin{verbatim}
      IMPLICIT DOUBLE PRECISION (A-H,O-Z)
\end{verbatim}

\noindent means that all variables (and functions) whose names start with any of the letters A, B, ..., H or
O, P, ... Z would have the type \verb+DOUBLE PRECISION+, unless explicitly given a different type.

\subsection{Special comments: D}
Sometimes in old code you can see a character "D" in the first column. This is considered a comment line,
unless you specify a compile option like \verb+-d-lines+ for the Intel Fortran compilers. Not all compilers
support this, gfortran does not seem to have an option for it. The effect of specifying the relevant
option is that the line is no longer considered a comment line, but is actual code. If supported by the compiler,
it is only available for fixed form source.

It is a rather awkward form of "preprocessing":
\begin{itemize}
\item
It is a compiler extension or at least not supported by all compilers. The now deprecated g77 compiler acknowledged
the existence of such lines, but provided no facilities \cite{F77DLine}.
\item
It is not available for free form source.
\item
A practical problem is that it does not evolve with the code itself, as in the normal build process, these
lines are considered comments.
\end{itemize}

It is just as easy to use ordinary code to provide debugging facilities:
\begin{verbatim}
*234567890
      PROGRAM FIXEDDEBUG

D     WRITE(*,*) 'Note: in debugging mode ...'
      WRITE(*,*) 'Hello, world'

      END PROGRAM FIXEDDEBUG
\end{verbatim}

The program (see the file \verb+fixeddebug.f+) is not accepted by gfortran because of the D-line, but Intel Fortran
compiles it with and without the option \verb+-d-lines+, only producing different output.

You could make the intent clearer using something along these lines (or a modern equivalent):
\begin{verbatim}
      PROGRAM FIXEDDEBUG
      LOGICAL DEBUG
      PARAMETER (DEBUG = .FALSE.)

      IF (DEBUG) WRITE(*,*) 'Note: in debugging mode ...'
      WRITE(*,*) 'Hello, world'

      END PROGRAM FIXEDDEBUG
\end{verbatim}

With FORTRAN the \verb+debug+ parameter was best put in an include file to ensure that every program unit
used the same setting. With Fortran a module containing such a parameter will do.

\begin{quote}
\emph{Note:} If you use a \emph{parameter} instead of a \emph{variable}, many compilers will even eliminate
the debugging code from the resulting program, but the code is compiled.
\end{quote}


\subsection{Compiler directives}
Besides specific options, most if not all compilers have what are known as \emph{compiler directives}. These
take the form of a comment with a special format. As they are specific to the compiler, the precise form
depends on the compiler, as well as what you can do with them. There are also compiler directives that
are defined as a standard, like the \emph{OpenMP} directives.\footnote{Compiler directives have their
equivalent in C. There they are called pragmas and work in much the same way.}

Here is one example: an implicit way to redefine the default precision for real variables \cite{IntelRealDirective}:

\begin{verbatim}
C For the Intel Fortran compiler:
!DIR$ REAL:8
\end{verbatim}

This specifies the \verb+kind+ of real variables that are not declared with an explicit kind themselves.

Another example of a completely different nature:

\begin{verbatim}
C For the gfortran compiler:
!GCC$ unroll 10
\end{verbatim}

The gfortran compiler supports loop unrolling and with this directive you control how this is done.

Since these directives come with a specific compiler, you will need to check the documentation of
the program or, more generally perhaps, the favoured compiler to see what the meaning is. It is not
always entirely innocent: loop unrolling is an optimisation technique, but barring errors in the
compiler, it should not matter for the outcome of a program whether the directive has been adhered
to or not. This is different for the first directive -- the default precision may have a significant
effect!


\subsection{Separate compilation}
One of the objectives of the design of FORTRAN, which continues in today's standards, is that
source files can be compiled independently. This shoud be read as: it is not necessary to
compile the whole program in one step. In FORTRAN that was fairly easy: besides include statements
there is no actual dependency possible. So, a source file can be compiled without knowing anything
about other parts of the program.

In Fortran, there are dependencies beyond included files: if the program unit uses one or more
(non-intrinsic) modules, then the order of compilation is that first the source files containing
these modules must be compiled and then the source file with the using program unit.

The idea of separate compilation is a very powerful one, certainly in times when desk checking
the code was a necessary first step, because the actual compilation might be done during the night,
with you coming back in the morning to find out that you made a typo somewhere. Compiling source files
separately and storing the resulting object files in libraries helps to minimize the risks of
losing time.

But a side effect of this is that some programmers interpreted this as meaning: you should put all
subroutines and functions in separate files. This does have one advantage, namely that the object files
would only contain a single program unit and the linker would incorporate only the program units
that are actually called. The disadvantage is that you may have to combine a large number of
files into a library, because all the little subroutines and functions are in individual source files.

With the modules we have nowadays it is much easier to manage the routines. You can hide the ones
that are not intended for outside use (so their names cannot cause trouble) and you can store
the routines that functionally belong together in one and the same module, or organise the code
in submodules.

\section{Subroutines and functions}
Placeholder:
\begin{verbatim}
- array(*) versus array(:)
- array(10) as the starting point
- constants as actual arguments
- intent
- temporary arrays - non-contiguous arrays
- checking interfaces
- entry
- external, also notation "*tan"
- Specific names for functions like max() and sin()
- Alternate return
- array(1) instead of array(*)
- statement functions

\end{verbatim}

\section{Floating-point numbers}
Nowadays, most computers use the IEEE format for representing floating-point numbers.
The two main types you will encounter are single-precision reals, occupying four
bytes of memory, and double-precision reals, occupying eight bytes. Fortran of old,
including FORTRAN, favours single-precision -- without any decoration a literal
number like \verb+1.23456789+ is single-precision, whereas many other languages
use double-precision reals by default.

Back in the days before the IEEE standard was widely adopted \cite(IEEE}, reals were
represented in many different ways and also the arithmetic operations we normally take
for granted were not fully portable. This led to all manner of complications if
you wanted to make your program portable, and that was and is certainly a goal of Fortran.

\subsection{REAL*4 and REAL*8}
One of the many extensions that were added by compiler vendors was the use of an asterisk~(*)
to indicate the precision:
\begin{verbatim}
*
*     Single precision real
*
      real*4 x
*
*     Double precision real
*
      real*8 y
*
*     Single precision complex
*
      complex*8 z
\end{verbatim}
The number indicates the number of bytes that a real would occupy. This has never been
part of a FORTRAN or Fortran standard. The \emph{kind} feature in Fortran is much more
flexible, as it can capture aspects of the representation of floating-point numbers
beyond mere storage.\footnote{The current standard defines a general model for representing
real numbers. This encompasses the IEEE formats, but is in fact more general.}

\begin{quote}
\emph{Note:} Such notation is sometimes used for integers and logicals as well. Again
the \emph{kind} feature is much more useful than merely indicating the storage size.
\end{quote}

\begin{quote}
\emph{Note:} I have used a Convex computer in the distant past that actually had two
different types of floating-point numbers that both were single-precision. One was
structured according to the IEEE standard and the other was a native format. The difference
for all intents and purposes was the interpretation of the exponent. The native format
was said to be a bit faster, but they occupied the same storage, four bytes.

For the same bit pattern the \emph{values} differed by a factor~4.
\end{quote}

\subsection{Literals in the source code}
One thing to keep in mind: if a literal number occurs in the source code, it is interpreted
as it appears, independent of the context. For Fortran this has been standardised: an expression
on the right-hand side is evaluated independently of the left-hand side. More concretely:
\begin{verbatim}
    double precision pi = 3.14159265358979323846264338327950288419716939937510
\end{verbatim}
\noindent may look to specify $\pi$ in some 50 decimals, but to the compiler it is
merely a slightly bizarre way of expressing it in \emph{single-precision}, so actually
only six or seven significant decimals. To get \emph{double-precision}, you need to
add a \emph{kind} or, as it was in FORTRAN, a "d" exponent (with some excess decimals removed):
\begin{verbatim}
    double precision pi = 3.141592653589793238462643d0
\end{verbatim}

You can see the difference if you run this program:
\begin{verbatim}
program diff_double_precision
    implicit none

    double precision, parameter :: pi_1 = 3.141592653589793238462643
    double precision, parameter :: pi_2 = 3.141592653589793238462643d0

    write(*,*) 'Difference: ', pi_1 - pi_2
end program diff_double_precision
\end{verbatim}
\noindent which prints (you may expect slight differences in the last few decimals with
different compilers):
\begin{verbatim}
 Difference:    8.7422780126189537E-008
\end{verbatim}

Some old FORTRAN compilers seem to have been less strict about the dichotomy between
the left-hand side and the right-hand side and would indeed interpret such literal
numbers as double-precision.

Another thing to keep in mind is that many compilers, both new and old, allow for
compiler options that turn the \emph{default} precision for a variable declared as \verb+real+
into \emph{double precision}. If a program relies on this behaviour, then you need to
carefully check the code.

\subsection{Input in the absence of a decimal point}
Disk storage nowadays is all but endless, but this luxury did not exist in the old days.
This may have been the reason for a little known or used feature in input of real numbers:
if a string representing a real number does not contain a decimal point, then the \emph{input~format}
may insert it.

Here is an example, using internal I/O to make it self-contained (see also the file \verb+input_no_point.f90+):
\begin{verbatim}
program show_insert_point
    implicit none

    real :: x
    character(len=10) :: string

    string = '1234'

    read( string, '(f4.0)' ) x
    write(*,*) x

    read( string, '(f4.2)' ) x
    write(*,*) x
end program show_insert_point
\end{verbatim}
It produces:
\begin{verbatim}
   1234.00000
   12.3400002
\end{verbatim}
With the format in the second read statement a decimal point is inserted!

It may have been useful in the past but it does suggest that to avoid surprises, you better not
use input format with a prescribed number of decimals.



\section{Control structures}
FORTRAN 77 came with a small number of control constructs and it was quite usual to construct
other control flows via \verb+IF+ and \verb+GOTO+ statements. It inherited some constructs
from its predecessors that are very uncommon nowadays: the arithmetic (or three-way) \verb+IF+ and the
computed \verb+GOTO+, as well as the \verb+ASSIGN+ statement. This part of the document
highlights these ancient idioms.

\subsection{Ordinary and nested DO-loops}
The ordinary \verb+DO+ loop in FORTRAN looks like this:
%
\begin{verbatim}
      DO 110 I = 1,10
          ... do something useful ...
  110 CONTINUE
\end{verbatim}
%
The statement label \verb+110+ indicates the end of the \verb+DO+ loop and anything in between
is repeatedly executed. The Fortran equivalent is, unsurprisingly:
%
\begin{verbatim}
do i = 1,10
    ... do something useful ...
enddo
\end{verbatim}
%
But there are a few more things to say about these \verb+DO+ loops. First of all, the
statement label needs not appear with a \verb+CONTINUE+ statement. It could very well be
put on the last executable statement:
%
\begin{verbatim}
      SUM = 0.0
      DO 110 I = 1,10
  110     SUM = SUM + ARRAY(I)
\end{verbatim}
%
It can even be used for multiple, nested, \verb+DO+ loops:
%
\begin{verbatim}
      SUM = 0.0
      DO 110 J = 1,10
      DO 110 I = 1,10
  110     SUM = SUM + ARRAY(I,J)
\end{verbatim}
%
To skip a part of the calculation, you can use a \verb+GOTO+ statement, where in Fortran
you would use a \verb+cycle+ or \verb+exit+ statement:
%
\begin{verbatim}
*
* Sum the positive elements only and only if the sum
* remains smaller than 1.0
*
      SUM = 0.0
      DO 110 I = 1,10
          IF ( ARRAY(I) .LE. 0.0 ) GOTO 110
          IF ( SUM .GT. 1.0 ) GOTO 120
          SUM = SUM + ARRAY(I)
  110 CONTINUE
  120 CONTINUE
\end{verbatim}
%
The example is a little contrived, so that you can see the use of the \verb+GOTO+ statement
for both \verb+cycle+ and \verb+exit+. The modern equivalent becomes:
%
\begin{verbatim}
!
! Sum the positive elements only and only if the sum
! remains smaller than 1.0
!
sum = 0.0
do i = 1,10
    if ( array(i) <= 0.0 ) cycle
    if ( sum > 1.0 ) exit
    sum = sum + array(i)
enddo
\end{verbatim}
%
Note that sharing statement labels in a nested \verb+DO+ loop makes it difficult to
see what a statement \verb+GOTO endlabel+ should mean: skip an iteration or skip
the rest of the inner \verb+DO+ loop:
%
\begin{verbatim}
      SUM = 0.0
      DO 110 J = 1,10
      DO 110 I = 1,10
          IF ( SUM .GT. 1.0 ) GOTO 110
  110     SUM = SUM + ARRAY(I,J)
\end{verbatim}
%
In FORTRAN 66 (also known as FORTRAN IV) there was a significant difference with
the \verb+DO+ loop you find in current Fortran: there were a large number of
constraints and as a curious interaction with the compilers not detecting that
these constraints were violated, quite often a \verb+DO+ loop would run at least once \cite{F66DoLoops}.
Consider this code:
%
\begin{verbatim}
* With the right compiler options, print this line once!
* (Note: FORTRAN 77, not strictly FORTRAN 66)
      DO 110 I = 1,0
          WRITE(*,*) 'FORTRAN 66: ', I
  110 CONTINUE
      WRITE(*,*) 'Current value of i:', i
\end{verbatim}
%
With FORTRAN 66 semantics the sample program (see \verb+f66_loop.f90+) prints:
%
\begin{verbatim}
 FORTRAN 66:            1
 Current value of i:           2
\end{verbatim}
With modern semantics it prints:
%
\begin{verbatim}
 Current value of i:           1
\end{verbatim}
%
This can result in subtle but nasty differences, if you are unaware of this
"feature",

As Ron Shepard explains on Fortran discourse \cite{F66Explanation}:

\begin{quote}
F66 do loops were not directly or specifically required to execute once, rather the
range parameters were restricted so that a single pass was always executed for conforming
code. The one-trip execution feature followed indirectly from these constraints. The loop
in the example code

\begin{verbatim}
    do i = 1,0
        write(*,*) 'FORTRAN 66: ', i
    enddo

    write(*,*) 'Current value of i:', i
\end{verbatim}

would have been illegal for five reasons. 1)~the do statement would have required an
integer statement label; the unlabeled do with enddo was not introduced until f90. 2)~the
termination value m2 is zero. In f66, the m1, m2, and m3 values were all required
to be greater than zero. 3)~the m1~value must be less than or equal to the m2~value,
a constraint on the programmer that is violated in this example. 4)~the enddo
statement did not exist in f66, it would have been a labeled continue statement. 5)~upon
execution of the loop, the loop control variable i is undefined, so the final write
statement is referencing an undefined integer.

Many compilers would not diagnose and detect violations to the m1, m2, m3 constraints,
so they would execute the loop with a single pass. But that was not required or
specified by the standard, the standard simply stated the requirements which were
violated by the programmers in these cases.

Of course, the write statements also violate f66 in other ways. 1)~Character constants
did not exist until f77. 2) List-directed i/o did not exist until f77. 3) The use of * to
specify the default output logical unit was not defined until f77.
\end{quote}

Some compilers still provide an option to allow for the FORTRAN 66 semantics, \cite{F66DoLoops}
which includes this feature:\footnote{I could not find such a flag for the
gfortran compiler, but for Intel Fortran oneAPI it is -f66.}


\subsection{Simulating a DO-WHILE loop}
There was no explicit \verb+DO WHILE+ construct in FORTRAN, at least not
in the standard. Therefore you would need to simulate it using any of the
following methods:

\vspace{\baselineskip}
\noindent \emph{A} \verb+DO+ \emph{loop with a large upper bound:}
\begin{verbatim}
* Find the right line in a file
      DO 110 I = 1,10000000
          READ( 10, '(A)' ) LINE
          IF ( LINE(1:1) .NE. '*' ) THEN
              GOTO 120
          ENDIF
  110 CONTINUE
  120 CONTINUE
* Found the start of the information, proceed
      ...
\end{verbatim}

\vspace{\baselineskip}
\noindent \emph{A combination of statement labels and} \verb+GOTO+ \emph{-- check at the start:}
\begin{verbatim}
* Find the right line in a file
      READ( 10, '(A)' ) LINE
  110 CONTINUE
      IF ( LINE(1:1) .NE. '*' ) GOTO 120
      READ( 10, '(A)' ) LINE
      GOTO 110
  120 CONTINUE

* Found the start of the information, proceed
      ...
\end{verbatim}
\noindent (This example is a bit artificial to keep it in line with the other two,
but similar constructs with different processing definitely occur in practice!)

\vspace{\baselineskip}
\noindent \emph{A combination of statement labels and} \verb+GOTO+ \emph{-- check at the end:}
\begin{verbatim}
* Find the right line in a file
  110 CONTINUE
      READ( 10, '(A)' ) LINE
      IF ( LINE(1:1) .EQ. '*' ) GOTO 110

* Found the start of the information, proceed
      ...
\end{verbatim}

A modern equivalent would either use the \verb+DO WHILE+ loop or the unlimited
\verb+DO+ loop:
%
\begin{verbatim}
!
! Find the right line in a file
!
read( 10, '(a)' ) line
do while (line(1:1) == '*' )
    read( 10, '(a)' ) line
enddo

!
! Found the start of the information, proceed
!
...
\end{verbatim}
%
Or:
%
\begin{verbatim}
!
! Find the right line in a file
!
do
    read( 10, '(a)' ) line
    if (line(1:1) == '*' ) exit
enddo

!
! Found the start of the information, proceed
!
...
\end{verbatim}

The precise location of the check on the condition depends on what the purpose is and
whether you can actually check it at the start of the loop, as with a \verb+DO WHILE+,
or whether you require some preliminary calculation first. If you want to convert
old-style source code, beware that the logic may sometimes have to be reverted,
particularly if the condition comes at the end of the loop.


\subsection{Three-way IF statements and computed GOTOs}
Two types of statements that are quite alien to what you find in modern-day programming
languages are the three-way or artihmetic \verb+IF+ statement and the computed \verb+GOTO+ statement.
The latter could be used to simulate a \verb+select case+ construct, the first on the
other hand was, in modern eyes, an unusual predecessor of the \verb+IF ... ELSE ... ENDIF+
block.

A \emph{computed} \verb+GOTO+ \emph{statement} takes a list of statement labels and a single integer expression:
%
\begin{verbatim}
      GOTO (100, 200, 300) JMP
  100 CONTINUE
      WRITE(*,*) 'Jump: 1'
      GOTO 400
  200 CONTINUE
      WRITE(*,*) 'Jump: 2'
      GOTO 400
  300 CONTINUE
      WRITE(*,*) 'Jump: 3'
  400 CONTINUE
      ... the rest ...
\end{verbatim}
%
Depending on the value of this expression (the value of \verb+JMP+ in the above example,
the control would jump to the Nth label. If the value was zero or lower, the \verb+GOTO+
would not be executed and the program control would simply continue with the next statement.
This is the case too with a value that is larger than the number of statement labels.
(See as an illustration the source file \verb+computed_goto.f90+)

Since there is nothing special about the statement labels the control would jump to, you had
to make sure to jump somewhere else after the handling of each case. In the example that
is done by jumping to label 400.

The \verb+select case+ construct of Fortran is better behaved, as you do not have to
take care of jumping to the end yourself and it is possible to select the case via strings
as well as integer values or even ranges.

There is nothing particularly magic about the \emph{three-way} \verb+IF+ \emph{statement}.
But you need to know how it works:
%
\begin{verbatim}
      IF (IVALUE) 100, 200, 300
  100 CONTINUE
      WRITE(*,*) 'Value is negative - ', IVALUE
      GOTO 400
  200 CONTINUE
      WRITE(*,*) 'Value is zero - ', IVALUE
      GOTO 400
  300 CONTINUE
      WRITE(*,*) 'Value is positive - ', IVALUE
  400 CONTINUE
      ... the rest ...
\end{verbatim}
%
The action, a jump to one of the three statement labels, to be taken depends on the
\emph{sign} of the integer expression. Often two of the statement labels would be the same,
as two possibilities are more common than three. To see it in action, see the source
file \verb+three_way_if.f90+. Both statement types still exist in Fortran, or at least
in the compilers, to support old-style programs.\footnote{The arithmetic IF statement
was deleted from the Fortran 2018 standard, as gfortran will report. Intel Fortran
will tell you this when you specify the standard as f18 -- "-stand", defaulting to the
Fortran 2018 standard.}


\subsection{Jumping to the end}
\label{jumpingtoend}
The \verb+GOTO+ statement was and is also used to jump to a completely different
part of the program unit for reporting error conditions:
%
\begin{verbatim}
      SUBROUTINE PRSQRT( X )

      IF ( X .LT. 0.0 ) GOTO 900

      WRITE(*,*) 'Square root of X = ', X, ' is ', SQRT(X)
      RETURN

900   CONTINUE
      WRITE(*,*) 'X should not be negative - ', X
      STOP
      END
\end{verbatim}

Such statements can be gathered at the end of the program unit so as not to clotter
the code that deals with normal processing. If you want to avoid the \verb+GOTO+ statement,
then the modern \verb+BLOCK+ construct will help:
%
\begin{verbatim}
subroutine print_sqrt( x )
    real :: x


    normal: block
        if ( x < 0.0 ) exit normal

        write(*,*) 'square root of x = ', x, ' is ', sqrt(x)
        return
    end block normal
    !
    ! Error processing
    !
    errors: block
        write(*,*) 'x should not be negative - ', x
        stop
    end block errors

end subroutine print_sqrt
\end{verbatim}

\emph{Note:} the \verb+errors+ block is merely introduced to syntactically
distinguish the error handling from the normal handling. It does not influence
in this form the flow of control.

\emph{Note:} GOTO statements have been frowned upon for a long time \cite{GOTOHarmful}.
But when used in a disciplined and sparse manner, they can actually clarify the flow of control.
In programming, very few statements have absolute truth. (Well, except \verb+a = .true.+ of course.)


\subsection{The ASSIGN statement}
The uses of \verb+GOTO+ so far have all been static: the statement labels were fixed
in the code. While the \verb+GOTO+ statement is frowned upon and it can certainly make
the control flow difficult to follow when you do not use it in an orderly fashion, there
is a possibility to use "dynamic" labels, so that the \verb+GOTO+ effectively jumps to
a varying location. This is achieved by the \verb+ASSIGN+ statement. It is not often
used, as in most cases better and especially clearer constructs are possible, even
in FORTRAN, but here is one possible case:

\noindent Suppose you have to compute something complicated in a number of places in a
program unit, based on a large number of variables, so that using a subroutine or
a function is awkward, as it leads to a very long argument list. Nowadays you
can easily use an internal routine, but this was not the case with FORTRAN.
So, with the \verb+ASSIGN+ statement you could store a location to return to,
jump ahead to the complicated piece of code, jump back when done, and remain in
the same routine. Here is a simple example:
%
\begin{verbatim}
      SUBROUTINE( ICASE )
      SAVE
      A = 1.0
      B = 2.0
      C = 3.0
      IF ( ICASE .EQ. 1 ) ASSIGN 100 TO JMP
      IF ( ICASE .EQ. 2 ) ASSIGN 200 TO JMP
      IF ( ICASE .EQ. 3 ) ASSIGN 300 TO JMP
      GOTO 900

* Case 1: use the result in F
  100 CONTINUE
      WRITE(*,*) 'Case ', ICASE, 'value is ', F
      GOTO 400

* Case 2: use the result in G
  200 CONTINUE
      C = 4.0
      WRITE(*,*) 'Case ', ICASE, 'value is ', G
      GOTO 400

* Case 3: use the result in H
  300 CONTINUE
      WRITE(*,*) 'Case ', ICASE, 'value is ', H
      GOTO 400

* All done, continue
  400 CONTINUE
      WRITE(*,*) 'Done'
      RETURN

*
  900 CONTINUE
* We can use the local variables directly
      F = A + B + C
      G = A + B * C
      H = A * B + C

* We are done, so return to the "caller"
      GOTO JMP
      END
\end{verbatim}

It is a useless and contrived example, but it is only meant to illustrate how
a "dynamic" jump can be constructed -- the part after statement label 900
is actually independent of whatever happens above it. You can extend it with
new cases, without having to worry about the computational part.

\subsection{The DO-loop with a real index variable}
A peculiar feature of FORTRAN 77 was the introduction of a DO-loop gouverned
by a real variable. This means that the precise values of the variable might
influnce the number of iterations, for instance. A simple program to
illustrate this (see \verb+real_do.f90+):
\begin{verbatim}
      PROGRAM REAL_DO
      IMPLICIT NONE

      INTEGER I, N
      REAL    R

      DO 120 I = 1,10
          N = 0
          DO 110 R = 0.0, 1.0*I, 0.1*I
              WRITE(*,*) R
              N = N + 1
          ENDDO
          WRITE(*,*) 'Number of iterations: ', N
      ENDDO
      END
\end{verbatim}

The output of the program with the Intel Fortran (leaving out the values of \verb+R+):
\begin{verbatim}
 Number of iterations:           11
 Number of iterations:           11
 Number of iterations:           10
 Number of iterations:           11
 Number of iterations:           11
 Number of iterations:           10
 Number of iterations:           11
 Number of iterations:           11
 Number of iterations:           10
 Number of iterations:           11
\end{verbatim}

So, some DO-loops get 11 iterations, what you would expect with exact arithmetic
and others get only 10. If you run this program with the gfortran compiler the result
is a uniform 11 iterations -- apparently the loop is controlled in a different way!
Apart from the unpredictability of the number of iterations with one compiler, you
can also not rely on the portability of the program.

\section{Memory management}

In the decades leading up to the FORTRAN 77 standard, memory management was simple:
declare what memory you need via statically sized arrays and that is it. There was
no dynamic memory allocation, at least not in the FORTRAN language. It may surprise
you, but even the concept of an operating system that took care of the computer
was fairly new, as illustrated by a 1971 book by D.W. Barron, titled "Computer
Operating Systems". Quoting from the book's jacket:
\begin{quote}
As the operating system is becoming an important part of the software complex
accompanying a computer system. A large amount of knowledge about the subject
now exists, mainly in the form of papers in computer journals. It is thus time for
a book that coordinates what is known about operating systems.
\end{quote}

There are, however, a few aspects of FORTRAN that make the story a bit more
complicated: \verb+COMMON+ blocks, \verb+EQUIVALENCE+ and the \verb+SAVE+ statement.
All three will be discussed here.


\subsection{COMMON blocks}
Variables, be they scalars or arrays, are normally passed via argument lists between
program units (the main program, subroutines or functions). This is the immediately
visible part. But you can also pass variables via \verb+COMMON+ blocks. These
constitute a form of global \emph{memory}, but not of global \emph{variables}, as
a \verb+COMMON+ block merely allocates memory and the mapping of memory
locations onto variables is up to the program units themselves. For instance:
\begin{verbatim}
      SUBROUTINE SUB1
      COMMON /ABC/ X(10)
      ...
      END

      SUBROUTINE SUB2
      COMMON /ABC/ A(5), B(5)
      ...
      END
\end{verbatim}
The \verb+COMMON+ block \verb+/ABC/+ appears in two subroutines, but in subroutine
\verb+SUB1+ it is associated with the array \verb+X+ of 10 elements and in the
subroutine \verb+SUB2+ it is associated with two arrays, \verb+A+ and \verb+B+,
both having five elements. The memory is shared, so that if you set \verb+X(1)+
to, say, \verb+1.1+ in subroutine \verb+SUB1+, then on the next call to subroutine
\verb+SUB2+, the array element \verb+A(1)+ will have that same value, as they occupy
the same memory location.

\verb+COMMON+ blocks should have the same size in all locations in the program's
code where they occur. That is difficult to ensure, hence it was common (no pun intended)
to put the declaration of \verb+COMMON+ blocks in so-called include files. Each program
unit that needed to address the memory allocated via these \verb+COMMON+ blocks could
then use the \verb+INCLUDE+ statement to have the compiler insert the literal text
of that include file.\footnote{The INCLUDE statement was actually a common compiler
extension.}

There are various ways that \verb+COMMON+ blocks were used:
\begin{itemize}
\item
Variables in \verb+COMMON+ blocks are persistent. At least, that was a very common
occurrence. The rules in the FORTRAN standard are more complicated, but certainly
with the \verb+SAVE+ statement you can rely on these variables to retain the
values between calls to a routine.
\item
Often routines in a library have to cooperate: one routine is used to set options
and other routines do the actual work. By using one or more \verb+COMMON+ blocks
these options do not need to be passed around via the argument list.
\item
Together with \verb+EQUIVALENCE+ statements you could use the \verb+COMMON+ blocks
to share workspace. Remember: back in the days memory was much and much more precious
and scarce than it is now. So, defining work arrays \verb+WORK+ (of type real)
and \verb+IWORK+ (of type integer) and making them equivalent to each other, you
could save on memory, if these arrays are not used at the same time.
\end{itemize}
Nowadays, it is much easier to pass large amounts of essentially private data
around, simply define a suitable derived type. Also, it is easy to allocate
work arrays as you require them and release them again when done.

A special \verb+COMMON+ block was the so-called \emph{blank} \verb+COMMON+
block. It had no name and it did not have to be declared with the same size
in all parts of the program. In fact, on some systems it could be used as
a flexible reservoir of memory, in much the same way as you have the
heap nowadays. But this facility was an extension to the standard.


\subsection{The SAVE statement}
According to the FORTRAN standard a local variable in a function or
subroutine does not retain its value between calls, unless it has the
\verb+SAVE+ attribute:
%
\begin{verbatim}
      SUBROUTINE ACCUM( ADD )
*
*     Accumulate the counts
*
      INTEGER ADD

      INTEGER TOTAL
      SAVE    TOTAL
      DATA    TOTAL / 0 /

      TOTAL = TOTAL + ADD
      IF ( TOTAL > 100 ) THEN
          WRITE(*,*) 'Reached: ', TOTAL
      ENDIF
      END
\end{verbatim}

However, some implementations, notably on DOS/Windows, used static
storage for these local variables, which meant that the variables would
\emph{seemingly} retain their values, even without the \verb+SAVE+ statement.
If a program relied on this property and was ported to a different environment,
all manner of havoc could be raised.

\begin{quote}
\emph{Note:} I have actually had lively, but not necessarily pleasant, debates on
whether the behaviour either way was correct. Sometimes the unexpected
behaviour was claimed to indicate a compiler bug.
\end{quote}

\subsection{The initial values of (local) variables}
A feature related in a way to the \verb+SAVE+ statement is the fact that
in both FORTRAN and Fortran variables do not get a particular initial
value, unless they have the \verb+SAVE+ attribute, implicitly or explicitly.
With older compilers local variables may be stored in static memory and
quite often they may have an initial value of zero or whatever the
equivalent is for the variable's type, but that is in all cases simply
a random circumstance. \emph{Never assume that a variable that has not
been explicity given a value, has a particular value.}

You can set the initial value in FORTRAN via the \verb+DATA+ statement:
%
\begin{verbatim}
      LOGICAL FIRST
      DATA FIRST / .TRUE. /
\end{verbatim}

This means that at the first call to the subroutine holding this variable
\verb+FIRST+, it has the value \verb+.TRUE.+. You can later set it to
\verb+.FALSE.+ to indicate that the subroutine has been called at least once
before, so that no initialisation is needed anymore:
%
\begin{verbatim}
* Subroutine that sums the values we pass
      SUBROUTINE SUM( x )
      INTEGER X

      INTEGER TOTAL
      LOGICAL FIRST
      DATA FIRST / .TRUE. /

      IF ( FIRST ) THEN
          FIRST = .FALSE.
          TOTAL = 0
      ENDIF

      TOTAL = TOTAL + X
      END
\end{verbatim}
(JUst a variation on the previous example). This is not a very interesting routine,
but it illustrates a typical use.

\emph{To emphasize:} This type of initialisation is done so that the variables
in question have the designated value at the first call. If you change
the value, then they retain that new value. No reinitialisation occurs.
(Actually, the value is not set on the first call, but rather is part
of the data section of the program as a whole. There is no separate
assignment.)

The \verb+DATA+ statement is not executable, it normally appears
somewhere in the section that defines the variables, but it may occur
elsewhere -- most FORTRAN and Fortran compilers are not strict about it.

There is some peculiar syntax involved:
%
\begin{verbatim}
      REAL X(100)
      DATA X / 1.0, 98*0.0, 100.0 /
\end{verbatim}

You can repeat values in much the same way as with edit descriptors
in format statements: a count followed by an asterisk ("*") and the value
to be repeated. It is also possible to use implied do-loops:
%
\begin{verbatim}
      INTEGER I
      REAL X(100)

      DATA (X(I), I = 1,100) / 100*1.0 /
\end{verbatim}

While it is more usual to set the values together with the declaration
of a variable nowadays, like:
%
\begin{verbatim}
  integer :: i
  real    :: x(100) = [ (1.0, i = 1,size(x)) ]
\end{verbatim}
\noindent the \verb+DATA+ statement is more versatile, because it is
not necessary to set the values for an array in one single statement:
%
\begin{verbatim}
    integer :: i
    real    :: x(100)

    data (x(i), i = 1,50)   / 50*1.0 /
    data (x(i), i = 51,100) / 50*0.0 /
\end{verbatim}

So, this old-fashioned statement may have its uses still.

Another peculiarity: the \verb+DATA+ statement has effect on the
size of the object file:

TODO


\subsection{Initialising variables in COMMON blocks: BLOCK DATA}
The \verb+DATA+ statement plays an important role when it comes to initialising
variables in a \verb+COMMON+ block. Since the \verb+COMMON+ blocks may
appear in more than one subprogram (main program, subroutines, functions),
they cannot be initialised in the same way as ordinary variables: which
\verb+DATA+ statement should prevail, if several initialise the same
\verb+COMMON+ variables?

Thus enter the \verb+BLOCK DATA+ program unit!

It is the only way to initialise variables in a \verb+COMMON+ block and
it is special, because it is not executable. The peculiar consequence is
that you cannot put it in a library: there is no reference to it, unlike
with subroutines and functions, so it would never be loaded. Instead, you
will normally put in the same file as the main program or link against
its object file explicitly.

The general layout is:
%
\begin{verbatim}
      BLOCK DATA
      ... COMMON blocks ...
      ... DATA statements ...
      END
\end{verbatim}

Here is a small example of this effect:\footnote{I use the gfortran compiler
to illustrate such effects, but it would be similar with other compilers.
And since most if not all FORTRAN features are still supported in Fortran,
I also use free-form sources.}
%
\begin{verbatim}
gfortran -c block.f90
gfortran -o common1 common.f90 block.f90
gfortran -o common2 common.f90
\end{verbatim}

Both build commands succeed, despite the fact that one part of the program is missing
in the second one.

The \verb+block.f90+ source file is:
%
\begin{verbatim}
BLOCK DATA
COMMON /ABC/ X
DATA X /42/
END
\end{verbatim}

The \verb+common.f90+ source file is:
%
\begin{verbatim}
PROGRAM PRINTX
COMMON /ABC/ X
WRITE(*,*) 'Expected value of X = 42:'
WRITE(*,*) 'X = ', X
END
\end{verbatim}

Program \verb+common1+ prints the value 42, whereas the other
program prints 0. If the \verb+BLOCK DATA+ program unit had been
an ordinary program unit, the building of this version would have failed
on an unresolved symbol or the like.


\subsection{Work arrays}
In the old days you could encounter arguments to a routine that represented
such workspace. Usually you would have to declare the arrays to a size that
matches the problem at hand. Here is an example from the \emph{LAPACK} library
for linear algebra:
%
\begin{verbatim}
      SUBROUTINE DGELS( TRANS, M, N, NRHS, A, LDA, B, LDB, WORK, LWORK,
     $                  INFO )
*
*  -- LAPACK driver routine (version 3.2) --
*  -- LAPACK is a software package provided by Univ. of Tennessee,    --
*  -- Univ. of California Berkeley, Univ. of Colorado Denver and NAG Ltd..--
*     November 2006
*
*     .. Scalar Arguments ..
      CHARACTER          TRANS
      INTEGER            INFO, LDA, LDB, LWORK, M, N, NRHS
*     ..
*     .. Array Arguments ..
      DOUBLE PRECISION   A( LDA, * ), B( LDB, * ), WORK( * )
*     ..
\end{verbatim}

In this case, the argument \verb+WORK+ is a double-precision array of
size \verb+LWORK+. In the comments that document the use of this routine
the precise usage is described:
%
\begin{verbatim}
*  WORK    (workspace/output) DOUBLE PRECISION array, dimension (MAX(1,LWORK))
*          On exit, if INFO = 0, WORK(1) returns the optimal LWORK.
*
*  LWORK   (input) INTEGER
*          The dimension of the array WORK.
*          LWORK >= max( 1, MN + max( MN, NRHS ) ).
*          For optimal performance,
*          LWORK >= max( 1, MN + max( MN, NRHS )*NB ).
*          where MN = min(M,N) and NB is the optimum block size.
*
*          If LWORK = -1, then a workspace query is assumed; the routine
*          only calculates the optimal size of the WORK array, returns
*          this value as the first entry of the WORK array, and no error
*          message related to LWORK is issued by XERBLA.
\end{verbatim}

This means that for a particular case you can either use one of the formulae
or the special value \verb+-1+ for \verb+LWORK+ to obtain an optimal value.
The work array itself would still be a statically declared array.

Note that with the current features of Fortran the interface could be greatly
simplified:\footnote{Intentionally left in fixed form.}
%
\begin{verbatim}
      SUBROUTINE DGELS( TRANS, A, B, INFO )
*
*     .. Scalar Arguments ..
      CHARACTER, INTENT(IN) :: TRANS
      INTEGER, INTENT(OUT)  :: INFO
*     ..
*     .. Array Arguments ..
      DOUBLE PRECISION, INTENT(INOUT) ::  A(:,:), B(:,:)
*     ..
\end{verbatim}
%
\noindent provided the interface is made explicit via a module or an interface block.

\section{Input and output intricacies}
Placeholder:
\begin{verbatim}
- standard input and output
- LU-numbers 5 and 6 (and 7)
- command-line arguments for file names
- big-endian and little-endian
- unformatted versus binary files
- list-directed input and output - also: /
- narrow formats (?)
- use of d00 in input
- FORTRAN 66 semantics: OPEN - STATUS = 'NEW' as default.
- Effect of BLANK = 'ZERO' versus BLANK = 'NULL'
- read(10,*) n, (array(i), i = 1,n)
- 32-bits machines and unformatted files
- direct-access files
- carriage control: 0, 1, +
\end{verbatim}

Almost any program will read some kind of input files and produce some kind of
output files. FORTRAN defined, roughly, three types of files:
\begin{itemize}
\item
Text files meant to read or edited by humans. These are known as \emph{formatted files}.
\item
Binary files with a record structure of sorts, so that you could read a part of the
record and then automatically jump to the next. These are \emph{unformatted, sequential files}.
They are compact and might be considered \emph{binary} files.
\item
Binary files with records that have a fixed length and where you can position the \verb+READ+
or \verb+WRITE+ action to a particular record: the \emph{unformatted direct-access files}.\footnote{
Actually, there are also formatted direct-access files, but these are seldom used.}
\end{itemize}

Input and output in FORTRAN was always oriented towards records. For instance, if you read a number
from a formatted file, then the read position is automatically move to the start of the next line,
independent of the amount of data left on the previous line.

Similarly, writing to a file always produced a complete line. And the next write action would start
on a new line.

Since Fortran 2003 the language also supports \emph{stream-access} for both unformatted and formatted
files. Before that standard, there were several more or less popular extensions to achieve the same
effect.


\subsection{Direct-access files}
By default, direct-access files are unformatted -- values are dumped to the file or retrieved
without the help of a human-readable format. Properties of direct-access files:
\begin{itemize}
\item
You have to specify a record length, which holds for all records, when opening such a file.
You can, however, open the file with different record lengths, if your application benefits
from that. In other words, the length is not a property of the file itself.
\item
Direct-access files can be read or written by specifying a record number. Thus, like the name
implies, you can jump around in the file at will.
\item
The record length is usually specified in \emph{bytes} but some compilers, like Intel Fortran,
use \emph{words} as the unit. Where there was no way to know programmatically in FORTRAN what
size was meant, since Fortran 2003, the intrinsic module \verb+iso_fortran_env+ contains the parameter \verb+file_storage_unit+
which is the size in bits.
\end{itemize}

Direct-access files, due to their simplicity, are compact and portable, as the structure of
unformatted files depends on the compiler that was used for building the program (see below).

The main issues that makes these files non-portables are the binary representation of the
numbers they contain. Nowadays, the main variation is the \emph{endianness}: the order of the
bytes that make up the number.

Here is a simple example of opening, writing and reading a direct-access file:

\begin{verbatim}
      PROGRAM DIRECTACCESS

      REAL VALUE(10)
      INTEGER I, REC

      OPEN( 10, FILE = 'directaccess.bin', ACCESS = 'DIRECT',
     &      RECL = 4*10 )

*
* CALCULATE SOME DATA AND WRITE THEM TO THE FILE
*
      DO 120 REC = 1,10
          DO 110 I = 1,10
              VALUE(I) = I + REC * 10.0
  110     CONTINUE

          WRITE( 10, REC = REC ) VALUE
  120 CONTINUE

*
* READ THE DATA - IN REVERSE ORDER
*
      DO 220 REC = 10,1,-1
          READ( 10, REC = REC ) VALUE(1), VALUE(2)
          WRITE( *, * ) VALUE(1), VALUE(2)
  220 CONTINUE
      END PROGRAM
\end{verbatim}

It produces output like:
\begin{verbatim}
   101.000000       102.000000
   91.0000000       92.0000000
   81.0000000       82.0000000
   71.0000000       72.0000000
   61.0000000       62.0000000
   51.0000000       52.0000000
   41.0000000       42.0000000
   31.0000000       32.0000000
   21.0000000       22.0000000
   11.0000000       12.0000000
\end{verbatim}


\subsection{Unformatted sequential files}
Data in sequential files, as the name suggests, are accessed in the order in which they appear in the files.
For unformatted files you write the records one by one and you can read the records back one by one.
But you cannot read more data from the record than its length. This is actually encoded in the file.
The following program will therefore fail, as it tries to read more data than present in the first
record:

\begin{verbatim}
      PROGRAM SEQUNFORM

      REAL VALUE(10)
      INTEGER I, J

      OPEN( 10, FILE = 'sequnform.bin', FORM = 'UNFORMATTED' )

*
* CALCULATE SOME DATA AND WRITE THEM TO THE FILE
* THE RECORDS GET LONGER
*
      DO 120 J = 1,10
          DO 110 I = 1,10
              VALUE(I) = I + J * 10.0
  110     CONTINUE

          WRITE( 10 ) (VALUE(I), I = 1,J )
  120 CONTINUE

*
* READ THE DATA - IN REVERSE ORDER
*
      REWIND( 10 )

*
* FIRST RECORD: ONLY ONE VALUE
*
      READ( 10 ) VALUE(1)
      WRITE( *, * ) VALUE(1)

*
* SECOND RECORD: TWO VALUES, BUT READ TEN
*
      READ( 10 ) VALUE
      WRITE( *, * ) VALUE(1)

      END PROGRAM
\end{verbatim}

Reading the first record works, but it fails on the second, as the program
tries to read 10 values, whereas the record only contain two (the gfortran was used):

\begin{verbatim}
   11.0000000
At line 38 of file sequnform.f (unit = 10, file = 'sequnform.bin')
Fortran runtime error: I/O past end of record on unformatted file

Error termination. Backtrace:

Could not print backtrace: libbacktrace could not find executable to open
#0  0xa11301fa
#1  0xa11278a1
#2  0xa1122d90
#3  0xa1148e3e
#4  0xa1130e81
#5  0xa1101841
#6  0xa11018e4
#7  0xa11013bd
#8  0xa11014f5
#9  0xb56c7613
#10  0xb76a26a0
#11  0xffffffff
\end{verbatim}




\section{Subjects}
\begin{verbatim}
- array(*) versus array(:)
- array(10) as the starting point
- history of computers:
   - hardware
   - memory management
   - tools like source code control systems
   - connections between computers, Internet
- equivalence
- constants as actual arguments
- intent
- temporary arrays - non-contiguous arrays
- implicit types
- double precision versus kind
- checking interfaces
- separate compilations, the misunderstanding of one routine per file
- fixed form and spaces
- standard input and output
- LU-numbers 5 and 6 (and 7)
- command-line arguments for file names
- real do-variables
- entry
- statement functions
- six characters
- numerical binary representations versus IEEE (IBM, Cray, Convex)
- big-endian and little-endian
- double complex
- unformatted versus binary files
- list-directed input and output - also: /
- narrow formats (?)
- use of d00 in input
- Cray pointers
- uppercase/lowercase letters
- D as comment character
- external, also notation "*tan"
- FORTRAN 66 semantics: OPEN - STATUS = 'NEW' as default.
- Effect of BLANK = 'ZERO' versus BLANK = 'NULL'
- Specific names for functions like max() and sin()
- Alternate return
- Equating logicals: logical x, y; if ( x .eq. y ) then
\end{verbatim}


References to be added: IEEE 754 and Fortran 90 standard.

\end{document}
