\section{Some caveats for programmers coming from C-like languages}

Some features of FORTRAN and indeed Fortran may confuse programmers who are used to C-like
languages. This section is meant to at least point out a few such features. It is evidently far
from complete. Since many of these features are common to both FORTRAN and Fortran, I will use the
word Fortran in this section to mean both versions.

\subsection{Terminology and semantics}
Fortran uses the term \emph{argument} to indicate the variables that are passed to and from
subprograms, where C uses the term \emph{parameter}. A \emph{parameter} in Fortran, however,
is a named constant. In a debugger it is most often not visible at all.

Another thing to watch out for is that a \emph{pointer} in Fortran is not merely a memory
address, as it is in C. A Fortran pointer holds far more information and it can only point
to variables that are either themselves pointers or have the \emph{target} attribute.

In C assignment, \verb+a = b+, is actually an operator. In Fortran, it is a statement and
therefore cannot be put into a combination like: \verb+a = b = c+.

\subsection{Subroutines and functions}
A function in Fortran always returns a value and you cannot ignore that, like in C. If \verb+f+
is a function (either in C or Fortran) returning an integer value, then:
\begin{verbatim}
     f( x ); /* Ignore the return value */
\end{verbatim}
\noindent is valid C, but in Fortran the return value always has to be handled.

The default method for passing variables to subroutines or functions in Fortran is \emph{by reference},
whereas in C it is \emph{by value}. That means, unless you specify the intent to be \verb+intent(in)+,
then you can always change the value of the actual variable.\footnote{Intent did not exist in FORTRAN.}

\subsection{Initialisation of variables}
In C code like:
\begin{verbatim}
     int f( int x ) {
         int y = 1;

         if ( x > 0 ) {
             y = 0;
         }
         return x + y;
     }
\end{verbatim}

is simply a short-hand for:
\begin{verbatim}
     int f( int x ) {
         int y ;

         y = 1;
         if ( x > 0 ) {
             y = 0;
         }
         return x + y;
     }
\end{verbatim}

\noindent that is, the variable \verb+y+ is set to 1 at the start of the function every time it is
called.

The Fortran code that \emph{looks} like exactly this:
\begin{verbatim}
     integer function f( x ) {
         integer, intent(in) :: x

         integer y = 1

         if ( x > 0 ) then
             y = 0
         endif
         f = x + y
     end function f
\end{verbatim}
\noindent behaves in a different way:

\begin{itemize}
\item
The local variable \verb+y+ has the value 1 at the start of the program and implicitly has the \verb+save+ attribute,
so it retains its value.
\item
It is \emph{not} reset to 1, unlike in the C version.
\item
If an argument \verb+x+ is passed with a positive value, then the variable \verb+y+ is reset to 0 and keeps that
value from there on. This is a side effect!
\end{itemize}

In fact the Fortran code is equivalent to:
\begin{verbatim}
     int f( int x ) {
         static int y = 1;

         if ( x > 0 ) {
             y = 0;
         }
         return x + y;
     }
\end{verbatim}

\subsection{Handling character strings}
C does not truly have character strings, it has arrays of characters. As C does not have intrinsic
features to keep track of the size of an array, it uses a convention to make sure that strings
are terminated -- the trailing NUL byte. A C program is itself responsible for maintaining this
NUL byte.

In contrast, Fortran uses strings of a defined length and pads strings with spaces. Thus:
\begin{verbatim}
     character(len=10) :: a

     a = '123'
\end{verbatim}

\noindent means that the \verb+a+ has the value '123_______' (where '_' is used to indicate a space).

This makes programming with strings a lot easier as you can simply rely on the compiler to make
sure that strings fit into the allotted memory:

\begin{verbatim}
     character(len=10) :: a, b
     character(len=15) :: c

     a = '1234567890'
     b = 'abcdefghij'

     c = a // b

     write(*,*) c
\end{verbatim}

\noindent will print the string '1234567890abcde', without any concern for the fact that \verb+a // b+
is longer than the variable \verb+c+ can hold.
