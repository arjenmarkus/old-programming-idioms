\section{Source forms}
Placeholder:
\begin{verbatim}
- separate compilations, the misunderstanding of one routine per file
- compiler directives (especially proprietary ones)
\end{verbatim}


Directives: just an example, taken from the thread on MAX/MIN on the Intel Fortran forum
\begin{verbatim}
module m_mod
  implicit none
  real, allocatable, dimension (:) :: aa, bb, cc
  !dir$ attributes align:32 :: aa, bb, cc
  !dir$ assume_aligned aa:32, bb:32, cc:32
end module m_mod
\end{verbatim}

The source form of FORTRAN, now known as \emph{fixed form}, shows its heritage in the era of punchcards:
Each line could be up to 80 characters long, but only the first 72 characters had actual meaning. Some
editors would add line numbers in the column 73--80 or people could add short comments. It had and has
a few curiosities beyond the mere signficance of the columns.

The alternative source form that was introduced with the Fortran~90 standard is more flexible and should
definitely be used for any new source.

\subsection{The significance of spaces}
The original fixed form for FORTRAN sources has at least the following curiosities:
\begin{itemize}
\item
In the columns 1 to 5 you can only put comments, if the first character is a comment character ("C" or "*"),
or statement labels.
\item
The sixth column is reserved to indicate that the previous line is continued:
\begin{verbatim}
      WRITE(*,*) 'A long sentence that spans over two',
     &           'or more lines'
\end{verbatim}
The continuation character, in the above the "\&", can actually be any character with the exception of
a zero (0). Some programmers would use digits, so that you can easily count the number of lines.
\item
The columns 7 to 72 were used for the actual code.
\item
The columns 73 to 80 were reserved for comments, to the discretion of the user.
\item
\emph{Spaces had no significance}, except within literal strings.
\end{itemize}
Especially this last property may come as a surprise. To illustrate this (see \verb+fixedform.f+):
\begin{verbatim}
      PROGRAM FIXEDFORM

      INTEGER :: I

      DO 100 I = 1.10
          WRITE(*,*) I
  100 CONTINUE
!
!     ILLUSTRATE THE FACT THAT SPACES HAVE NO MEANING ...
!
      DO 110 I = 1,10
      I F ( I . E Q . 4 ) T H E N
          W R I T E ( * , * ) I
      E N D I F
  110 C O N T I N U E
      END
\end{verbatim}
Keywords in FORTRAN and Fortran are not reserved words and in fixed form you can
put between the letters as many spaces you want, anywhere.

Running this program might produce the following output:
\begin{verbatim}
   578772136
           4
\end{verbatim}
"Might", because there is an uninitialised variable here: the line \verb+DO 100 I = 1.100+
contains a typo. Instead of a comma, it contains a decimal point and thus the line is
interpreted as:
\begin{verbatim}
      DO100I = 1.10
\end{verbatim}
The variable \verb+DO100I+ gets assigned the value 1.10, which is why there is no DO loop
to produce ten lines in the output, counting from 1 to 10. Such mistakes can be caught
by using the \verb+IMPLICIT NONE+ statement, a common enough extension.\cite{Lionel}\footnote{This statement
was formalized in "MIL-STD-1753", precisely to make the coding safer.}

\emph{Note to self: https://stevelionel.com/drfortran/2021/09/18/doctor-fortran-in-implicit-dissent/}


\subsection{The character set}
Officially, FORTRAN code should be written with capitals only. Lowercase letters are only
allowed in literal strings. Again, a heritage of the machinery of old. But it has been
allowed by compilers to use lowercase as well.

Beyond letters of the Latin alphabet you can use digits and underscores and several other characters.
Even today, the standard is quite strict: characters outside the officially supported set are only
allowed (or tolerated?) in literal strings.


\subsection{Variables, functions, names and types}
A feature that Fortran code relies on less and less is the use of \emph{implicit typing}: of old
a variable, or indeed a function, was of type \verb+integer+ if its name starts with one of: I, J, K, L, M or N.
If not, the variable or function was of type \verb+real+. You could declare the name as being of a different type, but
required an explicit declaration.

Because of this implicit typing, mistakes in names were a serious source of errors.

In FORTRAN, names were also limited to six (significant) characters, a limitation shared with the linkers.

This six-characters limit was even more severe as the names of subroutines, functions and \verb+COMMON+ blocks
were global for the whole program. There was no other scope or control of visibility. This made it
very important to explicitly define the names of such entities for any library you wrote, as naming conflicts
could not be solved via a renaming clause, as they can nowadays in Fortran.


\subsection{Special comments: D}
Sometimes in old code you can see a character "D" in the first column. This is considered a comment line,
unless you specify a compile option like \verb+/d-lines+ for the Intel Fortran compilers. Not all compilers
support this, gfortran does not seem to have an option for it. The effect of specifying the relevant
option is that the line is no longer considered a comment line, but is actual code. If supported by the compiler,
it is only available for fixed form source.

It is a rather awkward form of "preprocessing":
\begin{itemize}
\item
It is a compiler extension or at least not supported by all compilers. The now deprecated g77 compiler acknowledged
the existence of such lines, but provided no facilities.

\emph{Note to self: https://gcc.gnu.org/onlinedocs/gcc-3.4.6/g77/Debug-Line.html}
\item
It is not available for free form source.
\item
A practical problem is that it does not evolve with the code itself, as in the normal build process, these
lines are considered comments.
\end{itemize}

It is just as easy to use ordinary code to provide debugging facilities:
\begin{verbatim}
      PROGRAM FIXEDDEBUG

D     WRITE(*,*) 'Note: in debugging mode ...'
      WRITE(*,*) 'Hello, world'

      END PROGRAM FIXEDDEBUG
\end{verbatim}

The program (see the file \verb+fixeddebug.f+) is not accepted by gfortran because of the D-line, but Intel Fortran
compiles it with and without the option \ver+-d-lines+, only producing different output.

You could make the intent clearer using something along these lines (or a modern equivalent):
\begin{verbatim}
      PROGRAM FIXEDDEBUG
      LOGICAL DEBUG
      PARAMETER (DEBUG = .FALSE.)

      IF (DEBUG) WRITE(*,*) 'Note: in debugging mode ...'
      WRITE(*,*) 'Hello, world'

      END PROGRAM FIXEDDEBUG
\end{verbatim}

With FORTRAN the \verb+debug+ parameter was best put in an include file to ensure that every programn unit
used the same setting. With Fortran a module containing such a parameter will do.

\quote{
\emph{Note:} If you use a \emph{parameter} instead of a \emph{variable}, many compilers will even eliminate
the debugging code from the resulting program, but the code is compiled.}
