\section{Source forms}
The source form of FORTRAN, now known as \emph{fixed form}, shows its heritage in the era of punchcards:
Each line could be up to 80 characters long, but only the first 72 characters had actual meaning. Some
editors would add line numbers in the column 73--80 or people could add short comments. It had and has
a few curiosities beyond the mere signficance of the columns.

The alternative source form that was introduced with the Fortran~90 standard is more flexible and should
definitely be used for any new source.


\subsection{Fixed form versus free form}
FORTRAN source files usually have an extension \verb+.f+ and on PCs, before you could use long names,
the extension often was \verb+.for+. There is, however, nothing special about these file extensions. They
are merely a convention, useful for the compiler as it can use it to identify the source form. The
free form source is often indicated by the extension \verb+.f90+. \emph{This should not be taken to mean
that the code adheres to the Fortran~90 standard}, just as there is no particular difference as far as
file extensions are concerned for FORTRAN~66 or FORTRAN~77.\footnote{File extensions themselves are
a fairly recent feature of computer file systems. Older systems had different conventions.}

To elaborate a bit on these file extensions:
\begin{itemize}
\item
Fixed form is often identified by \verb+.f+ or for some compilers on Windows \verb+.for+, but
that is not mandated by any standard and compilers often support options to specify what the
source form is, if the extension is misleading.
\item
On file systems where file names are case-sensitive, such as Linux, an extension \verb+.F+ or \verb+.F90+
is often meant to automatically invoke a preprocessor like C's preprocessor. Alternatively, you
can also use compiler options to invoke the preprocessor explicitly.
\item
Preprocessing is not part of the FORTRAN or Fortran standards. It is simply an extension (no pun intended)
that may or may not be supported by the compiler. As a consequence, there is no prescribed syntax, though
very often the C conventions are used.
\end{itemize}


\subsection{The significance of spaces}
The original fixed form for FORTRAN sources has at least the following curiosities:
\begin{itemize}
\item
Lines are interpreted a comments if the character in column 1 is a "C" or a "*".
A common extension is the use of "D" to mark "debug" statements (see below).

From the Fortran standard onwards, an exclamation mark, "!", indicates the rest of the line
is a comment -- unless it appears inside a literal string.

(Some FORTRAN 66 compilers allowed the use of any non-numeric character in the first column
to mean a comment, but that is definitely not allowed since the FORTRAN 77 standard.)
\item
Columns 1 through 5 are reserved for numerical statement labels. Any positive number will do.
\item
Any non-blank character in the sixth column (other than "0") indicates that the previous line is continued:
\begin{verbatim}
*234567890
      WRITE(*,*) 'A long sentence that spans over two',
     &           'or more lines'
\end{verbatim}
The most common convention was to use the "\&". The "*" was not uncommon either.
Some programmers would use digits, so that you can easily count the number of lines.
\item
The columns 7 to 72 were used for the actual code.
\item
The columns 73 to 80 were reserved for comments, to the discretion of the user.
\item
\emph{Spaces had no significance}, except within literal strings.
\end{itemize}

Especially this last property may come as a surprise. To illustrate this (see \verb+fixedform.f+):
\begin{verbatim}
      PROGRAM FIXEDFORM

      INTEGER :: I

      DO 100 I = 1.10
          WRITE(*,*) I
  100 CONTINUE
!
!     ILLUSTRATE THE FACT THAT SPACES HAVE NO MEANING ...
!
      DO 110 I = 1,10
      I F ( I . E Q . 4 ) T H E N
          W R I T E ( * , * ) I
      E N D I F
  110 C O N T I N U E
      END
\end{verbatim}
Keywords in FORTRAN and Fortran are not reserved words and in fixed form you can
put between the letters as many spaces you want, anywhere.

Running this program might produce the following output:
\begin{verbatim}
   578772136
           4
\end{verbatim}
"Might", because there is an uninitialised variable here: the line \verb+DO 100 I = 1.100+
contains a typo. Instead of a comma, it contains a decimal point and thus the line is
interpreted as:
\begin{verbatim}
      DO100I = 1.10
\end{verbatim}
The variable \verb+DO100I+ gets assigned the value 1.10, which is why there is no DO loop
to produce ten lines in the output, counting from 1 to 10. Such mistakes can be caught
by using the \verb+IMPLICIT NONE+ statement, a common enough extension.\cite{drFortranImplicit}\footnote{This statement
was formalized in "MIL-STD-1753", precisely to make the coding safer.}


\subsection{The character set}
Officially, FORTRAN code should be written with capitals only. Lowercase letters are only
allowed in literal strings. Again, a heritage of the machinery of old. But it has been
allowed by compilers to use lowercase as well.

Beyond letters of the Latin alphabet you can use digits and underscores and several other characters.
Even today, the standard is quite strict: characters outside the officially supported set are only
allowed (or tolerated?) in literal strings.


\subsection{Variables, functions, names and types}
A feature that Fortran code relies on less and less is the use of \emph{implicit typing}: of old
a variable, or indeed a function, was of type \verb+integer+ if its name starts with one of: I, J, K, L, M or N.
If not, the variable or function was of type \verb+real+. You could declare the name as being of a different type, but
that required an explicit declaration.

Because of this implicit typing, mistakes in names were a serious source of errors.

In FORTRAN, names were also limited to six (significant) characters, a limitation shared with the linkers.

This six-characters limit was even more severe as the names of subroutines, functions and \verb+COMMON+ blocks
were global for the whole program. There was no other scope or control of visibility. This made it
very important to explicitly define the names of such entities for any library you wrote, as naming conflicts
could not be solved via a renaming clause, as they can nowadays in Fortran.

While the \verb+IMPLICIT+ statement is used nowadays mostly to turn off the implicit typing via \verb+implicit none+,
it can be used to define a particular type to variable whose names started with a particular character:
\begin{verbatim}
      IMPLICIT DOUBLE PRECISION (A-H,O-Z)
\end{verbatim}

\noindent means that all variables (and functions) whose names start with any of the letters A, B, ..., H or
O, P, ... Z would have the type \verb+DOUBLE PRECISION+, unless explicitly given a different type.

\subsection{Special comments: D}
Sometimes in old code you can see a character "D" in the first column. This is considered a comment line,
unless you specify a compile option like \verb+-d-lines+ for the Intel Fortran compilers. Not all compilers
support this, gfortran does not seem to have an option for it. The effect of specifying the relevant
option is that the line is no longer considered a comment line, but is actual code. If supported by the compiler,
it is only available for fixed form source.

It is a rather awkward form of "preprocessing":
\begin{itemize}
\item
It is a compiler extension or at least not supported by all compilers. The now deprecated g77 compiler acknowledged
the existence of such lines, but provided no facilities \cite{F77DLine}.
\item
It is not available for free form source.
\item
A practical problem is that it does not evolve with the code itself, as in the normal build process, these
lines are considered comments.
\end{itemize}

It is just as easy to use ordinary code to provide debugging facilities:
\begin{verbatim}
*234567890
      PROGRAM FIXEDDEBUG

D     WRITE(*,*) 'Note: in debugging mode ...'
      WRITE(*,*) 'Hello, world'

      END PROGRAM FIXEDDEBUG
\end{verbatim}

The program (see the file \verb+fixeddebug.f+) is not accepted by gfortran because of the D-line, but Intel Fortran
compiles it with and without the option \verb+-d-lines+, only producing different output.

You could make the intent clearer using something along these lines (or a modern equivalent):
\begin{verbatim}
      PROGRAM FIXEDDEBUG
      LOGICAL DEBUG
      PARAMETER (DEBUG = .FALSE.)

      IF (DEBUG) WRITE(*,*) 'Note: in debugging mode ...'
      WRITE(*,*) 'Hello, world'

      END PROGRAM FIXEDDEBUG
\end{verbatim}

With FORTRAN the \verb+debug+ parameter was best put in an include file to ensure that every program unit
used the same setting. With Fortran a module containing such a parameter will do.

\begin{quote}
\emph{Note:} If you use a \emph{parameter} instead of a \emph{variable}, many compilers will even eliminate
the debugging code from the resulting program, but the code is compiled.
\end{quote}


\subsection{Compiler directives}
Besides specific options, most if not all compilers have what are known as \emph{compiler directives}. These
take the form of a comment with a special format. As they are specific to the compiler, the precise form
depends on the compiler, as well as what you can do with them. There are also compiler directives that
are defined as a standard, like the \emph{OpenMP} directives.\footnote{Compiler directives have their
equivalent in C. There they are called pragmas and work in much the same way.}

Here is one example: an implicit way to redefine the default precision for real variables \cite{IntelRealDirective}:

\begin{verbatim}
C For the Intel Fortran compiler:
!DIR$ REAL:8
\end{verbatim}

This specifies the \verb+kind+ of real variables that are not declared with an explicit kind themselves.

Another example of a completely different nature:

\begin{verbatim}
C For the gfortran compiler:
!GCC$ unroll 10
\end{verbatim}

The gfortran compiler supports loop unrolling and with this directive you control how this is done.

Since these directives come with a specific compiler, you will need to check the documentation of
the program or, more generally perhaps, the favoured compiler to see what the meaning is. It is not
always entirely innocent: loop unrolling is an optimisation technique, but barring errors in the
compiler, it should not matter for the outcome of a program whether the directive has been adhered
to or not. This is different for the first directive -- the default precision may have a significant
effect!


\subsection{Separate compilation}
One of the objectives of the design of FORTRAN, which continues in today's standards, is that
source files can be compiled independently. This shoud be read as: it is not necessary to
compile the whole program in one step. In FORTRAN that was fairly easy: besides include statements
there is no actual dependency possible. So, a source file can be compiled without knowing anything
about other parts of the program.

In Fortran, there are dependencies beyond included files: if the program unit uses one or more
(non-intrinsic) modules, then the order of compilation is that first the source files containing
these modules must be compiled and then the source file with the using program unit.

The idea of separate compilation is a very powerful one, certainly in times when desk checking
the code was a necessary first step, because the actual compilation might be done during the night,
with you coming back in the morning to find out that you made a typo somewhere. Compiling source files
separately and storing the resulting object files in libraries helps to minimize the risks of
losing time.

But a side effect of this is that some programmers interpreted this as meaning: you should put all
subroutines and functions in separate files. This does have one advantage, namely that the object files
would only contain a single program unit and the linker would incorporate only the program units
that are actually called. The disadvantage is that you may have to combine a large number of
files into a library, because all the little subroutines and functions are in individual source files.

With the modules we have nowadays it is much easier to manage the routines. You can hide the ones
that are not intended for outside use (so their names cannot cause trouble) and you can store
the routines that functionally belong together in one and the same module, or organise the code
in submodules. Alternatively, to split a large source file into smaller pieces, you can use the INCLUDE statement.
